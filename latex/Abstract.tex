% Chapter Template
 \addcontentsline{toc}{chapter}{Abstract}
\chapter*{Abstract}% Main chapter title
 
\label{} % Change X to a consecutive number; for referencing this chapter elsewhere, use \ref{ChapterX}


\tab Recognition using biometrics has become a very demanding area in recent years, in terms of performance and ease of use. Diversity of applications requiring persons identification leads to find more suitable solutions adapted to field's requirements.
\\ \tab Particularly, there are multimodal biometric systems that combine several biometric modalities in order to take advantages of each modality. And like any other biometric system, multimodal biometric systems pass through the same following steps of recognition: extraction, matching and decision, with a fusion stage that can be executed in one of the precedent steps. In each one, there are numerous methods and algorithms, each one has its own advantages and disadvantages.
\\ \tab Therefore, many dependent choices are produced when we are searching for solution with good results. Decisions should be made mutually which results increase in freedom's degree in finding a suitable solution for a problem, hence, increasing in its complexity.
\\ \tab The goal of this project is the implementation of a tests platform destined not only for unimodal biometric systems but also for multimodal ones that combine fingerprint and palmprint, by implementing several methods used in different steps of recognition process of fingerprint and palmprint. Our platform will allow researchers to test different combinations of different methods with different configurations, in the same environment to enable them to evaluate these methods.
\\ \tab This platform will be extensible to allow researchers to add other methods and other modalities. Also, it will provide an easy to use interface for researchers. \\
\tab \textbf{Keywords:} biometrics, multimodality, modality, fingerprint, palmprint, person recognition, test, platform.
