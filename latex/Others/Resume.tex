% Chapter Template
 \addcontentsline{toc}{chapter}{Résumé}
\chapter*{Résumé}% Main chapter title
\label{Chapter2} % Change X to a consecutive number; for referencing this chapter elsewhere, use \ref{ChapterX}

\tab La reconnaissance par la biométrie devient un domaine exigeant ces dernières années, en termes de performances et d’aisance d’usage. Les applications qui nécessitent l’identification des individus, étant diversifiées, mobilisent le recours à des solutions adaptées aux exigences du domaine.\\
\tab Particulièrement, les systèmes biométriques multimodaux, qui combinent entre plusieurs modalités biométriques afin de profiter des avantages de chacune, sont une des solutions prometteuses. Comme tout autre système biométrique, le processus de reconnaissance basé sur plusieurs modalités passe par plusieurs étapes : extraction, appariement, décision, plus une étape de fusion qui peut être faite au niveau d’une des étapes précédentes. Pour chaque étape, il existe une panoplie de méthodes et d’algorithmes, chacune présentant des avantages ainsi que des insuffisances.
\\ \tab Donc, les choix à établir sont nombreux et ne sont pas indépendants si on vise une solution offrant de bons résultats de reconnaissance. Une plateforme supportant les différents paramètres à établir est donc nécessaire afin de tester une multitude de systèmes et ainsi permettre une comparaison fiable des performances sous les mêmes conditions de test. \\
\tab L’objectif de ce projet est d’implémenter une plateforme qui permet de tester non seulement des systèmes unimodaux, mais aussi multimodaux qui combinent l’empreinte digitale et l’empreinte palmaire, à travers l’implémentation de plusieurs méthodes intervenantes dans les différentes étapes du processus de reconnaissance de l’empreinte digitale et de l’empreinte palmaire. Cette plateforme va permettre aux chercheurs, de tester les différentes combinaisons des différentes méthodes, avec différentes configurations, sous le même environnement afin de pouvoir les évaluer. Cette plateforme va être extensible, ce qui permet l’ajout d’autres méthodes et d’autres modalités tout en assurant une interface facilement manipulable par les chercheurs. \\
\tab \textbf{Mot clés :} biométrie, multimodalité, modalité, reconnaissance des individus, empreinte digitale, empreinte palmaire, test, plateforme.


