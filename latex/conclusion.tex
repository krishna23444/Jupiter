\chapter*{Conclusion générale et perspectives}
\tab La biométrie multimodale est considérée comme la technique de reconnaissance des individus la plus sure grâce à son efficacité, performance et résistance aux attaques. A partir de ce constat, le domaine de recherche en biométrie est très actif et riche en termes de méthodes de reconnaissance biométrique de toute modalité. Cette richesse a impliqué la nécessité d’un outil de test qui permet aux chercheurs en biométrie multimodale d’évaluer leurs systèmes biométriques unimodaux et multimodaux. \\ \tab
Ce projet vient pour répondre à la problématique posée en offrant « Jupiter » une plateforme de test de systèmes biométriques unimodaux et la fusion multimodale, cette plateforme en premier lieu permet le test des méthodes des deux modalités d’empreinte digitale et d’empreinte palmaire écrite en Matlab, avec la possibilité d’étendre la plateforme en d’autres modalités et d’autres langages. \\ \tab
A travers notre plateforme, le chercheur en biométrie trouve plusieurs méthodes et bases de données de tests d’empreinte digitale et d’empreinte palmaires prêtes, il peut en ajouter des autres, tester un système biométrique de reconnaissance d’empreinte digitale, d’empreinte palmaire, de plus, tester la fusion multimodale au niveau des caractéristiques ou bien au niveau score, ainsi, il peut composer plus de 60 systèmes biométriques unimodaux et multimodaux prêts à tester. \\ \tab
« Jupiter » avec ses fonctionnalités, son extensibilité et le fait d’être un outil open source, constitue une solution parfaite pour la communauté des chercheurs en biométrie pour partager leurs tests, s’ils le veulent, donc « Jupiter » sera un espace de tests biométriques et toutes les informations concernant un test : le chercheur qui a lancé le est, les bases de données de test utilisées, la date, les résultats et les informations concernant l’environnement du test, un chercheur peut donc éviter de retester un système biométrique s’il trouve le test qu’il veut, ou bien le tester avec d’autres bases de données de tests ou effectuer le test sous un autre environnement et comparer ses résultats avec d’autres résultats de tests. \\ \tab
Afin de construire « Jupiter » telle qu’elle est maintenant, des choix ont été prises tout au long de notre travail :
\begin{enumerate}
\item Le choix des modalités à combiner : qui sont l’empreinte digitale et l’empreinte palmaire, effectué après une étude sur la biométrie et comparaison entre les modalités, nous avons sélectionné ses deux modalités à cause de leurs efficacité, performance, acceptabilité par les individus et leur universalité.

	\item Le choix des méthodes d’empreintes digitales et d’empreintes palmaires à implémenter : était basé sur l’étude faite pendant la phase d'état de l’art.
\item Les choix technologiques : pour l’architecture et les langages de programmation, sélectionnés d’une manière à garantir l’extensibilité de la solution, la réutilisabilité du code  et la réduction du temps de réponse vu le volume des BDDs de test pour renvoyer les résultats du test au chercheur rapidement.
\end{enumerate}
Aussi, pour comprendre le besoin des chercheurs en biométrie multimodale et d’y répondre, notre travail s’articulait autour de axes :
\begin{itemize}
	\item La compréhension des processus de reconnaissance d’empreinte digitale, d’empreinte palmaire et la fusion multimodale.
	\item Concevoir et réaliser une plateforme de test facilement manipulable par les chercheurs, qui concerne principalement les empreinte digitales et palmaires en garantissant l’extensibilité de la solution.
\end{itemize}
Malgré que « Jupiter » réponde aux objectifs qui lui ont été fixé, elle peut être amélioré, c’est pour cela, nous proposons les perspectives suivantes :
\begin{itemize}
	\item Ajouter d’autres modalités comme le visage, l’iris, …etc.
	\item Étendre « Jupiter » pour qu’elle supporte d’autres représentations d’empreinte digitale (représentation en texture) et d’empreinte palmaire (représentation en minuties) aussi que les images de hautes résolutions.

	\item Enrichir la plateforme avec d’autres méthodes de reconnaissance et d’autres bases de données de tests.
	\item Développer le test de fusion multimodale en ajoutant les autres niveaux de fusion : niveau capteurs et niveau décision.
    \item Permettre aux chercheurs d’écrire les méthodes en d’autres langages que Matlab, comme python, C++, …etc.
    \item Ajouter la possibilité de communication entre les chercheurs sur la plateforme et de créer des groupes de partage (laboratoire, équipe d’un projet, binôme ..etc).

    \item Donner la possibilité aux chercheurs de tester les méthodes de classification des bases de données biométriques.
\end{itemize}











