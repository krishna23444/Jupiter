% Chapter Template


 \addcontentsline{toc}{chapter}{ \AR {ملخص}}
 
\begin{otherlanguage}{arabic}

\chapter*{ \AR {ملخص}}


\label{Chapter4} 

 \tab أصبح التعرف على هوية الأفراد من خلال القياسات الحيوية مجالا متطلبا في السنوات الأخيرة، من حيث الأداء وسهولة الاستخدام. تنوع مجالات التطبيقات التي تحتاج الى تحديد هوية الأشخاص أدى الى اللجوء الى حلول ملائمة لمتطلبات هذا الميدان. \\
\tabوعلى وجه الخصوص، هناك نظم التحقق من الهوية التي تجمع بين عدة طرائق للقياس الحيوي للسماح بالاستفادة من ميزات كل طريقة، وكأي نظام قياس حيوي لتحديد الهوية، يمر نظام التحقق من الهوية متعدد الوسائط للقياس الحيوي بنفس خطوات عملية التعرف التالية: الاستخراج، المطابقة، القرار بالإضافة إلى مرحلة الدمج التي تنفذ على مستوى إحدى الخطوات السابق ذكرها. في كل مرحلة توجد عدة طرق وخوارزميات، لكل منها مزاياها وعيوبها.\\
\tab مما يجعل الخيارات متعددة، ومرتبطة بعضها البعض عند البحث عن نظام قياس حيوي ذو نتائج جيدة. فعند اتخاذ القرارات يجب أخذ هذا الارتباط في عين الاعتبار مما يؤدي إلى زيادة في درجة حرية مشكلة إيجاد حل مناسب وبالتالي زيادة تعقيده.\\
\tab الهدف من هذا المشروع هو إنجاز منصة تسمح ليس فقط باختبار النظم الأحادية الواسطة ولكن أيضا المتعددة الوسائط التي تجمع بين بصمة الأصبع وبصمة راحة اليد، من خلال تنفيذ عدد من الطرق والخوارزميات التي تتدخل في المراحل المختلفة من عملية التعرف على بصمة الأصبع وبصمة راحة اليد. هذه المنصة تسمح للباحثين أيضا لاختبار تركيبات مختلفة من طرق التعرف، مع إمكانية ضبط إعداداته وهذا في نفس البيئة، و ذلك من أجل التمكن من تقييمها بشكل أفضل. هذه المنصة ستكون قابلة للتطور لكي تتيح إمكانية إضافة طرق تعرف ووسائط قياسات حيوية أخرى، وستوفر واجهة مستخدم تتيح للباحثين التعامل معها بسهولة. \vspace{10px}\\\tab
الكلمات المفتاحية: القياسات الحيوية، تعدد الوسائط، واسطة، التعرف على الأفراد، بصمة الأصبع، بصمة راحة اليد، اختبار، منصة.

 			
\end{otherlanguage}