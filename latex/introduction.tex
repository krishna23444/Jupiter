% Chapter Template
\chapter*{Introduction}% Main chapter title
\lhead{INTRODUCTION}
\tab Aujourd'hui, avec l'évolution technologique et l'informatisation des différentes activités, le contrôle d'accès aux données est devenu un aspect primordial, et sa sécurisation peut être si importante que l'activité elle-même, ceci nous mène à parler de la reconnaissance des individus qui est un axe de recherche en plein développement. Ce contexte rend les méthodes traditionnelles de reconnaissance des individus insuffisantes comme le mot de passe qui peut être compromis par un tiers, ou la puce électronique qui peut être volée. Cela conduit à utiliser des moyens permettant d'identifier les individus d'une manière fiable et performante en assurant un service important de la sécurité qui est la non-répudiation. La biométrie qui est la reconnaissance des individus à partir de leurs traits physiques distinctifs, s'est imposée ces dernières années dans plusieurs applications comme l'accès aux établissements, la connexion aux réseaux informatiques, le pointage du personnel …etc., dans différents domaines comme le commerce, la médecine et le gouvernement. Plusieurs modalités biométriques comme l'empreinte digitale, l'iris, la signature, la marche, l'ADN et autres sont appliquées, et le choix de modalité s'effectue en fonction de l'application à développer et le niveau de sécurité exigé. \\ \tab
 

Néanmoins, l'utilisation d'une seule modalité souffre de plusieurs insuffisances vu qu'il n'existe pas de modalité satisfaisant toutes les caractéristiques qui qualifient une modalité parfaite comme la sensibilité des données biométriques à certains bruits différents d'une modalité à l'autre, la non-universalité, la sensibilité aux attaques . Le recours à d'autres alternatives devient donc nécessaire. La biométrie multimodale, une forme de la multi-biométrie combinant plusieurs modalités, est une solution permettant de bénéficier des points forts et remédier aux insuffisances de chacune. Cela permet d'offrir des systèmes biométriques plus efficaces et plus fiables. \\ \tab
Plusieurs chercheurs se sont intéressés aux multitudes de modalités et de formes biométriques. Par conséquent, plusieurs méthodes et algorithmes intervenant dans les différentes étapes du processus de reconnaissance, ont apparu. Pour obtenir et étudier l'utilité et l'efficacité d'une méthode, les chercheurs ont besoin de les tester soit par eux-mêmes ou bien en les envoyant à des plateformes dédiées aux tests des méthodes biométriques selon la disponibilité de ces dernières. Cependant, le chercheur perd beaucoup de temps et d'effort pendant l'implémentation du test et s'il opte pour le test à distance, il doit attendre pour qu'il reçoit les résultats. De plus, la comparaison des résultats obtenus à partir de plusieurs plateformes, restant comme des boites noires, devient biaisée vu que ces dernières peuvent ne pas préciser le protocole de test suivi, ou ne pas correspondre aux tests effectués par le chercheur de manière autonome et différente.  \\ \tab

L'objectif de notre projet open source « Jupiter » est de concevoir et de réaliser une plateforme de test des systèmes biométriques multimodaux, qui permet le test des systèmes unimodaux et leur fusion afin de donner un système multimodal. Notre étude s'intéresse plus précisément aux méthodes de reconnaissance de deux modalités qui sont l'empreinte digitale, l'empreinte palmaire et leur fusion, avec la possibilité d'étendre la plateforme vers d'autres modalités. La plateforme contient à la base un ensemble de méthodes prêtes à tester et permet aux chercheurs d'ajouter d'autres méthodes d'empreinte digitale, d'empreinte palmaire ou de leur fusion. \\ \tab

Ce mémoire est organisé en sept chapitres répartis sur deux parties, la première partie avec ses trois chapitres constituent notre état de l'art, où le premier chapitre abordera les généralités en biométrie, la multi-biométrie et la multimodalité, le deuxième chapitre étudie la reconnaissance d'empreinte digitale et ses principales méthodes, le troisième chapitre présente la reconnaissance d'empreinte palmaire et ses principales méthodes. Nous enchaînons avec la deuxième partie qui présente notre contribution, cette partie englobe quatre chapitres. Le premier chapitre présente le contexte de projet et expose l'analyse de besoins, le deuxième chapitre montre la conception de Jupiter. Ensuite, le troisième chapitre comporte une présentation de la plateforme, les choix technologiques et les méthodes implémentées et le dernier chapitre les résultats de tests effectués. Enfin, nous clôturons ce rapport avec une conclusion et des perspectives du travail.
