\documentclass[a4paper]{report}

%====================== PACKAGES ======================
\usepackage[utf8]{inputenc}
\usepackage[T1]{fontenc}
\usepackage[french]{babel}
\usepackage{amsmath}
\usepackage[colorinlistoftodos]{todonotes}
\usepackage{url}
\usepackage{multicol}
%Refs
\usepackage{caption}

\usepackage[authoryear]{natbib}


%pour les informations sur un document compilé en PDF et les liens externes / internes
\usepackage{hyperref}
%pour la mise en page des tableaux


\usepackage{enumitem}
\usepackage{float}%pour gérer les positionnement d'images
\usepackage{array}
\usepackage{booktabs}
\usepackage{multirow}
\usepackage{lscape}
\usepackage{amssymb}
\usepackage{pifont}
\usepackage{tabularx}
\usepackage{longtable}
%Images and figures
\usepackage{setspace}
\usepackage{graphicx}
\setcounter{secnumdepth}{3}

\usepackage{hyperref}
\hypersetup{
	colorlinks,
	citecolor=blue,
	filecolor=black,
	linkcolor=black,
	urlcolor=black
}

% for termes
\usepackage{tgtermes}
%pour utiliser \floatbarrier
%\usepackage{placeins}
%\usepackage{floatrow}
%espacement entre les lignes
\usepackage{setspace}

%modifier la mise en page de l'abstract
\usepackage{abstract}
%police et mise en page (marges) du document

%Mis en page :
\usepackage[top=2cm, bottom=2cm, left=2cm, right=2cm]{geometry}
%Footer and Header
\usepackage{fancyhdr}
\setlength{\headheight}{13.1pt}
\pagestyle{fancyplain}
\lhead{}
\cfoot{}
\fancyfoot[R]{\thepage}
\renewcommand{\baselinestretch}{1.5}

%Pour les galerie d'images
\usepackage{subfig}
%Page de garde
\usepackage{framed}


%Abbreviations
\usepackage[french]{nomencl}
\makenomenclature
%Rotating
 \usepackage{rotating}
 \usepackage{tikz}
\newcommand*\rot{\rotatebox{90}}
%\hypersetup{urlcolor=black, colorlinks=bleu}  Colors hyperlinks in blue - change to black if annoyingv`	
%====================== INFORMATION ET REGLES ======================

%rajouter les numérotation pour les \paragraphe et \subparagraphe
\setcounter{secnumdepth}{4}
\setcounter{tocdepth}{4}

\hypersetup{							% Information sur le document
pdfauthor = {KEBAILI Zohra Kaouter,
		KECHIDA Fatima Zahra,
		},			% Auteurs
pdftitle = {Plateforme de test multicritères adapté aux solutions
	multi-biométriques},			% Titre du document
pdfsubject = {Mémoire de Projet},		% Sujet
pdfkeywords = {Tag1, Tag2, Tag3, ...},	% Mots-clefs
pdfstartview={FitH}}					% ajuste la page à la largueur de l'écran
%pdfcreator = {MikTeX},% Logiciel qui a crée le document
%pdfproducer = {}} % Société avec produit le logiciel
%======================== DEBUT DU DOCUMENT ========================
\renewcommand{\baselinestretch}{1.3}
\renewcommand{\thetable}{\Roman{table}}
\captionsetup[table]{name=Tableau}
\begin{document} 
%page de garde
%====================== INCLUSION DES PARTIES ======================

\begin{titlepage}
 \begin{center}
 \includegraphics[scale=0.9]{Resources/entete.png}\\
 \vspace*{1cm}
  \LARGE
  \textbf{Mémoire de fin d’études\\}
  \large
 \textbf{ Pour l’obtention du diplôme d’ingénieur d’état en Informatique}\\
  \LARGE
  	\vspace{2cm}
  \textbf{Option: Systèmes et ingénierie de logiciels}\\
  \vspace{1cm}
  \LARGE
  \textbf{Thème}\\
  \vspace{1cm}
  \LARGE
  \setlength{\fboxsep}{0.5cm}
  \begin{framed}
	\textbf{Plateforme de test multicritères adapté aux solutions multi-biométriques}
  \end{framed}
  \vspace{2cm}
  \begin{table}[H]
   \setlength{\tabcolsep}{2cm}
    \large
	\centering
	\begin{tabular}{ll}
		\textbf{Réalisé par :}    
		 & \textbf{Encadré par : } \\  \\
		 -\textsc{ Kebaili} Zohra Kouater 
	
	& -\textsc{ Benatchba} Karima  \\
		-\textsc{ Kechida} Fatima Zahra 
    &-\textsc{ Artabaz} Saliha  

	\end{tabular}
  \end{table}
  \vspace{\fill}
  \large
  \textbf{Promotion 2016/2017}
        
 \end{center}
\end{titlepage}
\pagenumbering{roman}
\setcounter{page}{2}
\cleardoublepage
\addcontentsline{toc}{chapter}{\contentsname}
\tableofcontents
\cleardoublepage
\addcontentsline{toc}{chapter}{\listfigurename}
\listoffigures
\cleardoublepage
\addcontentsline{toc}{chapter}{\listtablename}
\listoftables
\cleardoublepage
\chapter*{Liste des sigles et abréviations}% Main chapter title
\begin{table}[H]
	\label{my-label}
	\begin{tabular}{ll}
	    \textbf{ACI}  & Analyse en Composantes Indépendantes     \\
	    \textbf{ACP}  & Analyse en Composantes Principales     \\
	     \textbf{ACP à noyaux}  & Analyse en Composantes Principales à noyaux     \\
	    \textbf{ACP2D}  & Analyse en Composantes Principales en deux dimensions     \\
	    \textbf{ADL}  & Analyse Discriminante Linéaire     \\
	    \textbf{BDD}  & Base De Données    \\
		\textbf{CN}   & Crossing Number                          \\
	  \textbf{DCT}  & Discrete Cosine transform                 \\
		\textbf{DET}  & Detection Error Tradeoff                 \\
		\textbf{EER}  & Equal Error Rate                         \\
     	\textbf{FMR}  & False Match Rate                     \\
		\textbf{FNMR} & False Non-Match Rate                     \\
		\textbf{k-NN} & k- Nearest Neighbor                    \\
		\textbf{MCC}  & Minutia Cylinder-Code     \\
		\textbf{OM}   & Orientations Map                         \\
		\textbf{PIN}  & Personal Identification Number                \\
		\textbf{ROC}  & Receiver Operating Characteristic        \\
		\textbf{RoI}  & Region of Interest                       \\
		\textbf{SDK}  & Software Development Kit                 \\
	    \textbf{SIFT}  & Scale-Invariant Feature Transform                \\

		\textbf{SP}   & Singulier Points      \\       
		\textbf{SVM}  &  Machines à Vecteurs de Support     \\  
		\textbf{TR}  &  Taux de Reconnaissance                    \\  

	\end{tabular}
\end{table}
\pagenumbering{arabic}

% Chapter Template
\chapter*{Introduction}% Main chapter title
\lhead{INTRODUCTION}
\tab Aujourd'hui, avec l'évolution technologique et l'informatisation des différentes activités, le contrôle d'accès aux données est devenu un aspect primordial, et sa sécurisation peut être si importante que l'activité elle-même, ceci nous mène à parler de la reconnaissance des individus qui est un axe de recherche en plein développement. Ce contexte rend les méthodes traditionnelles de reconnaissance des individus insuffisantes comme le mot de passe qui peut être compromis par un tiers, ou la puce électronique qui peut être volée. Cela conduit à utiliser des moyens permettant d'identifier les individus d'une manière fiable et performante en assurant un service important de la sécurité qui est la non-répudiation. La biométrie qui est la reconnaissance des individus à partir de leurs traits physiques distinctifs, s'est imposée ces dernières années dans plusieurs applications comme l'accès aux établissements, la connexion aux réseaux informatiques, le pointage du personnel …etc., dans différents domaines comme le commerce, la médecine et le gouvernement. Plusieurs modalités biométriques comme l'empreinte digitale, l'iris, la signature, la marche, l'ADN et autres sont appliquées, et le choix de modalité s'effectue en fonction de l'application à développer et le niveau de sécurité exigé. \\ \tab
 

Néanmoins, l'utilisation d'une seule modalité souffre de plusieurs insuffisances vu qu'il n'existe pas de modalité satisfaisant toutes les caractéristiques qui qualifient une modalité parfaite comme la sensibilité des données biométriques à certains bruits différents d'une modalité à l'autre, la non-universalité, la sensibilité aux attaques . Le recours à d'autres alternatives devient donc nécessaire. La biométrie multimodale, une forme de la multi-biométrie combinant plusieurs modalités, est une solution permettant de bénéficier des points forts et remédier aux insuffisances de chacune. Cela permet d'offrir des systèmes biométriques plus efficaces et plus fiables. \\ \tab
Plusieurs chercheurs se sont intéressés aux multitudes de modalités et de formes biométriques. Par conséquent, plusieurs méthodes et algorithmes intervenant dans les différentes étapes du processus de reconnaissance, ont apparu. Pour obtenir et étudier l'utilité et l'efficacité d'une méthode, les chercheurs ont besoin de les tester soit par eux-mêmes ou bien en les envoyant à des plateformes dédiées aux tests des méthodes biométriques selon la disponibilité de ces dernières. Cependant, le chercheur perd beaucoup de temps et d'effort pendant l'implémentation du test et s'il opte pour le test à distance, il doit attendre pour qu'il reçoit les résultats. De plus, la comparaison des résultats obtenus à partir de plusieurs plateformes, restant comme des boites noires, devient biaisée vu que ces dernières peuvent ne pas préciser le protocole de test suivi, ou ne pas correspondre aux tests effectués par le chercheur de manière autonome et différente.  \\ \tab

L'objectif de notre projet open source « Jupiter » est de concevoir et de réaliser une plateforme de test des systèmes biométriques multimodaux, qui permet le test des systèmes unimodaux et leur fusion afin de donner un système multimodal. Notre étude s'intéresse plus précisément aux méthodes de reconnaissance de deux modalités qui sont l'empreinte digitale, l'empreinte palmaire et leur fusion, avec la possibilité d'étendre la plateforme vers d'autres modalités. La plateforme contient à la base un ensemble de méthodes prêtes à tester et permet aux chercheurs d'ajouter d'autres méthodes d'empreinte digitale, d'empreinte palmaire ou de leur fusion. \\ \tab

Ce mémoire est organisé en sept chapitres répartis sur deux parties, la première partie avec ses trois chapitres constituent notre état de l'art, où le premier chapitre abordera les généralités en biométrie, la multi-biométrie et la multimodalité, le deuxième chapitre étudie la reconnaissance d'empreinte digitale et ses principales méthodes, le troisième chapitre présente la reconnaissance d'empreinte palmaire et ses principales méthodes. Nous enchaînons avec la deuxième partie qui présente notre contribution, cette partie englobe quatre chapitres. Le premier chapitre présente le contexte de projet et expose l'analyse de besoins, le deuxième chapitre montre la conception de Jupiter. Ensuite, le troisième chapitre comporte une présentation de la plateforme, les choix technologiques et les méthodes implémentées et le dernier chapitre les résultats de tests effectués. Enfin, nous clôturons ce rapport avec une conclusion et des perspectives du travail.

\pagestyle{fancy}
\part{Synthèse bibliographique}
\label{part1}
\lhead{GENERALITES SUR LA BIOMETRIE}
%\epigraph{I recall seeing a package to make quotes}{Snowball}
\chapter{Généralités sur la biométrie}
\label{Chapter1} % Change X to a consecutive number; for referencing this chapter elsewhere, use \ref{ChapterX}

\section{Introduction}
\tab Les méthodes traditionnelles utilisées pour authentifier un individu se basaient sur une connaissance « knowledge-based » (exemple : les mots de passe) ou sur une possession « token-based » (exemple : les badges, la pièce d'identité, les clés, etc.). Cependant, ces deux méthodes ont leurs inconvénients, tels que le risque d'oublier le mot de passe ou être deviné par tiers ou encore perdre le badge.
Une alternative pratique et sécurisée pour répondre à ces problèmes est l'utilisation de la biométrie \citep{Perronnin2002} qui consiste à identifier une personne à partir de ses caractéristiques physiques, comportementales ou biologiques.
Dans ce chapitre, nous allons nous intéresser à des généralités sur la biométrie, nous allons présenter les modalités biométriques, les systèmes biométriques, les domaines d'application de la biométrie et la multi-biométrie, nous présentons la définition de la multi-biométrie, ses formes et nous détaillons sa forme multimodale.
\section{Modalités biométriques} 
Les caractéristiques biométriques par lesquelles il est possible de vérifier l’identité d’un individu sont appelées modalités biométriques. Ces modalités sont classées en trois catégories :
\begin{itemize}
	\item \textbf{Modalités physiques : }
	se basent sur la reconnaissance des différents traits physiques particuliers, qui sont permanents et uniques pour toute personne (empreinte digitale, visage, etc.).
	\item \textbf{Modalités biologiques : }
	se basent sur l’analyse des données biologiques liées à l’individu (ex : ADN, le salive, l'odeur, l'analyse du sang de différents signaux physiologiques, ainsi que la fréquence cardiaque ou EEG, etc.).
	\item \textbf{Modalités comportementales : }
	se basent sur l’analyse des comportements d’un individu (ex : la dynamique de frappe au clavier, la reconnaissance vocale, la reconnaissance dynamique des signatures, la démarche, etc.).
\end{itemize}
\vspace{1cm} 
Pratiquement, pour qu'une caractéristique humaine soit considérée comme une caractéristique biométrique il faut qu'elle satisfasse les exigences suivantes \citep{Ross} :
\begin{itemize}
	\item \textbf{Universalité : }tous les individus à identifier doivent posséder cette caractéristique.
	\item 
	\textbf{Unicité : }les caractéristiques doivent être suffisamment distinctes d'un individu de la population à un autre.
	\item 
	\textbf{Permanence : }elle doit être suffisamment invariante sur une période de temps.
	\item \textbf{Mesurabilité : }elle doit être mesurable quantitativement.
\end{itemize}
Du point de vue de système, les propriétés suivantes doivent être également prises en compte \citep{Ross} :
\begin{itemize}
	\item \textbf{Performance : }la précision de reconnaissance requise dans une application doit être réalisable en utilisant les caractéristiques.
	\item 
	\textbf{Acceptabilité : }désigne la volonté du sujet (l’individu) de présenter ses caractéristiques biométriques.
	\item 
	\textbf{Résistance aux attaques : }il s'agit de la difficulté d'utiliser des caractéristiques biométriques falsifiées (par exemple, des fausses empreintes digitales dans le cas des modalités physiologiques et de mimétisme dans le cas d'une modalités comportementales).
	
\end{itemize}
\vspace{1cm} 
Le tableau \ref{chap1:tab1} compare entre les modalités biométriques selon les propriétés citées précédemment :
\begin{longtable}{|p{2cm}|c|c|c|c|c|c|c|} 
	\hline
	\rot{{\textbf{\begin{tabular}[c]{@{}l@{}}Modalité\\ biométrique\end{tabular}}}}&
	\rot{\textbf{Universalité}}& \rot{\textbf{Unicité} }& \rot{\textbf{Permanence} }& \rot{\textbf{Mesurabilité}}& \rot{\textbf{Performance}}& \rot{\textbf{Acceptabilité} }& \rot{\textbf{\begin{tabular}[c]{@{}l@{}}Résistance aux\\ attaques\end{tabular}}} \\ \hline
	ADN&Elevée&Elevée&Elevée&Faible&Elevée&Faible&Faible \\ \hline
	Oreille&Moyenne&Moyenne&Elevée&Moyenne&Moyenne&Elevée&Moyenne \\ \hline
	Visage&Elevée&Faible&Moyenne&Elevée&Faible&Elevée&Elevée \\ \hline
	Thermo gramme du visage&Elevée&Elevée&Faible&Elevée&Moyenne&Elevée&Faible \\ \hline
	Empreinte digitale&Moyenne&Elevée&Elevée&Moyenne&Elevée&Moyenne&Moyenne \\ \hline
	Démarche&Moyenne&Faible&Faible&Elevée&Faible&Elevée&Moyenne \\ \hline
	Géométrie de la main&Moyenne&Moyenne&Moyenne&Elevée&Moyenne&Moyenne&Moyenne \\ \hline
	Veine de la main&Moyenne&Moyenne&Moyenne&Moyenne&Moyenne&Moyenne&Faible \\ \hline
	Iris&Elevée&Elevée&Elevée&Moyenne&Elevée&Faible&Faible \\ \hline
	Frappe de touche&Faible&Faible&Faible&Moyenne&Faible&Moyenne&Moyenne \\ \hline
	Odeur&Elevée&Elevée&Elevée&Faible&Faible&Moyenne&Faible \\ \hline
	Empreinte palmaire&Moyenne&Elevée&Elevée&Moyenne&Elevée&Moyenne&Moyenne \\ \hline
	Rétine&Elevée&Elevée&Moyenne&Faible&Elevée&Faible&Faible \\ \hline
	Signature&Faible&Faible&Faible&Elevée&Faible&Elevée&Elevée \\ \hline
	Voix&Moyenne&Faible&Faible&Moyenne&Faible&Elevée&Elevée \\ \hline
	
	\caption{Comparaison entre les modalités biométriques \citep{Jain2004}.}
	\label{chap1:tab1}
\end{longtable}


\clearpage
\section{Domaines d’application de la biométrie}
La biométrie peut être employée dans un grand nombre d'applications. Elle peut aider à rendre les opérations, les transactions et la vie quotidienne plus sûres et plus pratiques. Selon \citep{Jain2004}, les domaines d’applications de la biométrie peuvent être divisés en trois groupes :
\begin{itemize}
	\item \textbf{Applications commerciales : }telles que la connexion au réseau informatique, la sécurité des données électroniques, l'e-commerce, l’accès à Internet, les guichets automatiques, les cartes de crédit, le contrôle d'accès physique, la gestion des dossiers médicaux, etc.
	\item \textbf{Applications gouvernementales : }telles que les cartes d'identité nationale, les permis de conduire, la sécurité sociale, l'aide sociale, le contrôle des frontières, le contrôle des passeports, etc.
	\item \textbf{Applications médico-légales : }par exemple, l’identification des cadavres, les enquêtes criminelles, l’identification des terroristes, les tests de paternité et l’identification des enfants disparus, etc.
\end{itemize}

\section{Systèmes biométriques}

Un système biométrique est un ensemble de composants matériels et de données. Les composants matériels sont les capteurs et les programmes de comparaison, de classification et etc. Et les données sont les modèles numériques qui permettent de gérer une modalité biométrique, à partir de l’étape de capture des informations biométriques des individus jusqu'à l’étape de prise de décision lors d’une tentative d’accès \citep{wayman2005introduction}. Dans cette section, nous allons présenter les processus fonctionnels d'un système biométrique ainsi que son architecture, ensuite, les mesures de performances d'un système biométrique, enfin, nous exposons ses limitations.
\subsection{Processus fonctionnels d’un système biométrique}
Les systèmes biométriques ont trois processus fonctionnels divisés en deux phases : une phase pour enrôler les modèles des individus de la population et une autre phase de reconnaissance\citep{Ross2004a}.

\subsubsection{Phase d’enrôlement}
Pendant cette première phase, l’individu est enregistré dans le système pour la première fois. Une ou plusieurs modalités biométriques sont capturées et enregistrées dans une base de données. Les données de la base sont les données non biométriques dites biographiques, comme le nom, le numéro de la carte d’identité nationale, etc. (voir la figure \ref{fig:chapitre1enrollement}) \citep{meyer2009}.

\begin{figure}[H]
	\centering
	\includegraphics[width=0.9\linewidth]{processusfonctionnels1}
	\caption{Exemple d'enrôlement d’une empreinte digitale d'individu dans un système biométrique \citep{meyer2009}.}
	\label{fig:chapitre1enrollement}
\end{figure}
\subsubsection{Phase de reconnaissance}
La deuxième phase fonctionnelle d’un système biométrique peut être une authentification ou identification selon l’application concernée.
\begin{itemize}
	\item En mode d’authentification, le système doit répondre à une question de type : « Suis-je bien la personne que je prétends être ? ». L’utilisateur propose une identité au système et qui doit vérifier que l’identité de l’individu est bien celle proposée, il suffit donc de comparer le modèle extrait de l’identité prétendue. Si ce modèle a déjà une occurrence dans la base de données avec le modèle extrait de l’individu au moment de capture lors de la tentative d’authentification, on parle alors de correspondance 1:1 (voir la figure \ref{fig:chapitre1authentification}) \citep{Perronnin2002}.
	\\ Prenant un exemple d'un individu X qui souhaite retirer de l’argent à un distributeur de billets, en entrant son code personnel d’identification (code PIN), et en présentant une modalité biométrique. Le système acquiert alors les données biométriques et va les comparer uniquement avec le modèle enregistré correspondant à X, pour décider si X est authentique ou imposteur \citep{meyer2009}.
	\begin{figure}[H]
		\centering
		\includegraphics[width=0.9\linewidth]{processusfonctionnels3}
		\caption{Exemple d'authentification d’une empreinte digitale d'individu dans un système biométrique \citep{meyer2009}.}
		\label{fig:chapitre1authentification}
	\end{figure}
	\item En mode d’identification, le système doit deviner l’identité de l’individu qui affirme implicitement qu’il est déjà enrôlé par le système. Il répond donc à une question de type : « Qui suis- je ? ». Dans ce mode, le système compare le modèle de l’individu avec les différentes occurrences de la base de données. On parle alors de correspondance 1: N.
	Le système biométrique va trouver l’identité de la personne dont le modèle possède le degré de similitude le plus élevé avec le modèle biométrique présenté en entrée lors de la tentative d’identification. Si le plus grand score de similarité du modèle biométrique présenté en entrée avec tous les modèles de la BDD est inférieur à un seuil minimum fixé, l’individu est rejeté. Ce qui implique que l’utilisateur n’était pas une des personnes enrôlées par le système. Dans le cas contraire, la personne est acceptée \citep{Perronnin2002} (voir la figure \ref{fig:chapitre1identification}).\\
	Nous pouvons citer comme exemple, l’accès à un bâtiment sécurisé, où tous les utilisateurs autorisés à y entrer sont enrôlés par le système. Lorsqu’un individu essaye de pénétrer dans le bâtiment, il doit d’abord présenter ses données biométriques au système, et selon la résultat de l'identification de l’identité de l’utilisateur, le système lui accorde ou non le droit d’entrer \citep{meyer2009}.
	\begin{figure}[H]
		\centering
		\includegraphics[width=0.7\linewidth]{processusfonctionnels2}
		\caption{Exemple d'authentification d’une empreinte digitale d'individu dans un système biométrique \citep{meyer2009}.}
		\label{fig:chapitre1identification}
	\end{figure}
	
\end{itemize}
\subsection{Architecture des systèmes biométriques}
Les systèmes biométriques diffèrent l'un des autres en fonction du matériel de capture de l'information biométrique utilisé, des technologies exploitées et des algorithmes appliqués. 
Dans ce qui suit, nous allons décrire la structure générale d’un système biométrique indépendamment de toute modalité, de tout matériel, de toute méthode et de toute technologie, donc un système biométrique qui est générique. Ce système biométrique dit génétique se compose principalement de cinq sous-systèmes (ou modules)\citep{iso2006iec}. La figure \ref{fig:archiofbiometricsys} illustre le flux d'informations dans un système biométrique, ses composants et ses différents modes de fonctionnements.
\begin{figure}[H]
	\centering
	\includegraphics[width=1\linewidth]{archiofbiometricsys}
	\caption{Architecture du système biométrique \citep{iso2006iec}.}
	\label{fig:archiofbiometricsys}
\end{figure}
\subsubsection{Sous système de collecte de données }
Ce système se charge de l’opération d’acquisition ou de capture de l'information biométrique. Les sous-systèmes de collecte de données se différencient par :
\begin{itemize}
	\item Le type du capteur utilisé suivant la modalité biométrique du système global.
	\item Ses caractéristiques techniques.
	\item La manière de présenter le signal d’entrée. Exemple : image d’empreinte digitale ou enregistrement vocal.
	\item Le processus de conversion du signal d’entrée en une forme standard qui peut être manipulée par un ordinateur.
	\item La nécessité de coopération de l’individu ou non. Par exemple, prendre une image faciale ou scanner l’empreinte digitale.
\end{itemize}
\subsubsection{Sous système de traitement du signal 	}
Ce système traite d'abord les données biométriques capturées pour garder uniquement les données pertinentes qui peuvent distinguer les individus, il procède comme suit:
\begin{itemize}
	\item Élimine le bruit des données en sortie du sous-système précédent,qui peut étre généré à cause de la qualité du capteur et la résolution de l'image, les conditions d'éclairage et la position de capture.
	\item Applique une segmentation en utilisant un modèle prédéfini de segmentation afin de faciliter la phase de reconnaissance.
	\item Pour améliorer la qualité du modèle et optimiser sa taille de stockage, une opération d’extraction des caractéristiques est effectuée sur les données biométriques.
\end{itemize}

\subsubsection{Sous système de stockage de données}
Ce sous système sauvegarde les modèles biométriques des individus enrôlés. A la demande du sous système de comparaison(expliqué ci-dessous), il récupére un ou plusieurs modèles biométriques pendant la phase reconnaissance (authentification ou identification). Il s'occupe aussi de la mise à jour du modèle biométrique après une authentification ou identification si le nouveau modèle acquis est de meilleure qualité par rapport à celui déjà enrôlé du même individu.
Les modèles biométriques peuvent être enregistrés avec leurs données non biométriques, dans des bases de données souvent séparées logiquement ou physiquement pour des raisons de sécurité, sur des cartes intelligentes ou des dispositifs comme un ordinateur ou un téléphone mobile.
\subsubsection{Sous système de comparaison}
Ce système compare entre deux modèles biométriques en entrée, et selon la similarité entre eux, il donne en sortie : un score \footnote{Les scores indiquent la correspondance entre le modèle d’acquisition de chaque modalité biométrique composant le système avec le modèle enrôlé.} en cas d'une authentification, et un ensemble de scores au cas d'une identification.
\subsubsection{Sous système de prise de décision}
À partir de(s) score(s) trouvé(s) dans le précédant sous système et d’une politique de décision l’individu sera accepté ou considéré comme imposteur.\\
La politique de décision peut :
\begin{itemize}
\item Rejeter l’identité proclamée de tout individu dont le modèle biométrique n’a pas été acquis (enrôlé).
\item Accepter l’identité enrôlée, si le score est supérieur à un seuil prédéfini et le considérer comme imposteur dans le cas contraire.
\item Accepter les modèles biométriques dont les scores sont inférieurs à un seuil qui dépend de \citep{davida2002infrastructure} :	
\begin{list}{$\bullet$}{} 
	\item \textbf{L’individu : }il y des individus qui possèdent des caractéristiques distinctives plus que d’autres individus. C’est pour cette raison que le système utilise un seuil préfixé qui dépend de la distinctivité des caractéristiques de l’individu. Une distinctivité élevée engendre un seuil élevé et vice versa tout en tenant en compte des paramètres de la sécurité du système,un système biométrique avec un niveau de sécurité bas, peut tolérer les fausses acceptations alors le seuil fixé sera plus grand qu'un système biométrique de niveau de sécurité élevé.
	\item \textbf{La transaction : }pour une même application on peut voir plusieurs fonctionnalités à des droits d’accès différents, on peut associer à chaque droit d’accès un seuil pour contrôler plus l’accès à une opération ou à une donnée.
	\item \textbf{Le contexte : }d’autres informations peuvent être prises en considération pour fixer un seuil variant, comme les moments habituels d’accès au système, quand la dernière tentative d’accès était faite.
\end{list}
\item Donner à tous les individus un nombre fixe de tentatives possibles pour retourner un ou plusieurs scores inférieurs au seuil.
\end{itemize}

\subsection{Performances des systèmes biométriques }
\label{performance}
Il existe plusieurs métriques pour mesurer les performances d'un système biométrique. Dans cette section, nous présentons les mesures des taux d'erreur et les courbes de performance.
\subsubsection{Mesures des taux d’erreur }
Selon \citep{mansfield2002best}, les mesures des taux d’erreur sont divisées en trois groupes : les taux d’erreur de correspondance, taux d’erreur d'acquisition d'images et taux d’erreur de décision.
\begin{itemize}	
	\item \textbf{Taux d'erreur de correspondance} 
	\begin{list}{$\bullet$}{} 
		\item \textbf{Taux de fausses correspondances\footnote{Appelé en Anglais False Matching Rate (FMR).} : } c'est le taux introduit par l’algorithme de comparaison, entre la donnée biométrique acquise d'un individu et un modèle correspondant à un autre individu \citep{Jain2004}. 
		\item \textbf{Taux de fausses non-correspondances\footnote{Appelé en Anglais False Non-Matching Rate (FNMR).} : }c'est le taux introduit par l’algorithme de comparaison entre la donnée biométrique acquise d'un individu et le modèle correspondant au même individu \citep{Jain2004}.
	\end{list}
	\item \textbf{Taux d'erreur d'acquisition d'images} 
	\begin{list}{$\bullet$}{} 
		\item \textbf{Taux d'échec à acquérir : } reflète les tentatives de vérification ou d’identification pour lesquelles le système biométrique n’a pas pu acquérir l’information biométrique causé par les défauts de matériel, l'absence de l'individu, les conditions environnementales ...etc \citep{mansfield2002best}.
		\item \textbf{Taux d'échec à enrôler : }indique la proportion des individus pour lesquels la donnée biométrique n’a pas pu être générée correctement durant la phase d’enrôlement. Par exemple le cas où les personnes n’ont pas d’empreintes pour des raisons génétiques, ou des empreintes quasi-inexistantes pour des raisons médicales \citep{mansfield2002best}.
	\end{list}
	\item \textbf{Taux d'erreur de décision} 
	Les deux erreurs qui peuvent se produire pendant la phase de décision sont le rejet des utilisateurs \textit{légitimes} ou l'acceptation des \textit{imposteurs}.
	\begin{list}{$\bullet$}{} 
		\item \textbf{Taux de fausse acceptation : }pourcentage des imposteurs acceptés par erreur. Il est calculé comme suit :
		\begin{center}
			\begin{equation}\label{eq:TFA}
			TFA=\dfrac{Nombre \; de\; fausses \;acceptations\; (imposteurs\; acceptes)}{Nombre\; de\; tentatives \;d'acces\; non\; legitimes}
			\end{equation}
		\end{center}
		\item \textbf{Taux de faux rejets : }pourcentage des utilisateurs légitimes rejetés par erreur. Il est calculé comme suit :
		\begin{center}
			\begin{equation}\label{eq:TFR}
			TFR=\dfrac{Nombre\; de\; faux\; rejets \;(utilisateurs\; legitimes)}{Nombre\; de \;tentatives\; d'acces\; legitimes}
			\end{equation}
		\end{center}
		\item \textbf{Taux d'égale erreur : }concerne le point où FMR et FNMR sont égaux. Ce taux est fréquemment utilisé pour donner un aperçu sur la performance d'un système biométrique. Plus la valeur de ce taux d'erreur est faible, plus la précision du système biométrique est élevée \citep{liu2001practical} (voir figure \ref{fig:chapitre1err}). 
		\begin{center}
			\begin{figure}[H]
				\centering
				\includegraphics[width=0.75\linewidth]{eer}
				\caption{Equal Error Rate (EER).}
				\label{fig:chapitre1err}
			\end{figure}
		\end{center}
	\item \textbf{Taux de reconnaissance :} indique le nombre de comparaisons non erronés, il est calculé comme suit : 1 - la valeur de FNMR dans le seuil où nous avons trouvé le ERR.
		\item \textbf{Taux d'identification : }le taux d’identification au rang r est la proportion de transactions d’identification, par des utilisateurs enrôlés dans le système, pour lesquels l’identifiant de l’utilisateur est dans les r identifiants retournés.
		\item \textbf{Taux de fausse-positive identifications : }la probabilité de retourner une liste non vide dans l'identification des utilisateurs non enrôlés.
		\item \textbf{Taux de fausse-négative identifications : }le pourcentage d'échec d'identification d'un individu enrôlé où l’identifiant de l'individu ne figure pas dans la liste des identifiants retournée.
		
	\end{list}
	
\end{itemize}
\subsubsection{Courbes de performance }
\begin{itemize}
	\item \textbf{Courbe ROC : }représente graphiquement la relation entre le taux de vrais rejets FRR et taux de fausses acceptations FAR pour des différentes valeurs du seuil de décision \citep{egan1975signal}.\\
(voir figure \ref{fig:chapitre1roc}). 
	\begin{center}
		\begin{figure}[H]
			\centering
			\fbox{\includegraphics[width=0.5\linewidth]{roc}}
			\caption{Exemple de la courbe (ROC) \citep{Mainguet2017}.}
			\label{fig:chapitre1roc}
		\end{figure}
	\end{center}
	\item \textbf{Courbe DET: }c'est par essence une courbe ROC dont on représente directement l’évolution d’un taux d’erreur en fonction d’un autre pour la rendre plus lisible et plus exploitable. Le seuil de décision doit être ajusté en fonction de l'application ciblée : haute sécurité, basse sécurité ou compromis entre les deux (voir figure \ref{fig:chapitre1det}). 
	\begin{center}
		\begin{figure}[H]
			\centering
			\fbox{\includegraphics[width=0.5\linewidth]{det}}
			\caption{Exemple La courbe (DET) \citep{Mainguet2017}.}
			\label{fig:chapitre1det}
		\end{figure}
	\end{center}
	\item \textbf{Courbe cumulative des correspondances CMC : }représente les valeurs du rang d’identification et les probabilités d’une identification correcte inférieure ou égale à ces valeurs (voir figure \ref{fig:chapitre1cmc}). 
	\begin{center}
		\begin{figure}[H]
			\centering
			\includegraphics[width=0.6\linewidth]{cmc}
			\caption{Courbe de caractéristiques cumulatives de correspondance montrant la performance du rang 1 au rang 15 \citep{Hu2015}.}
			\label{fig:chapitre1cmc}
		\end{figure}
	\end{center}
\end{itemize}

\subsection{Limites des systèmes biométriques}
\label{section:limitation}

Bien que les systèmes biométriques offrent une solution fiable pour la reconnaissance et en pratique, ces systèmes sont utilisés dans nombreux systèmes commerciaux, Ils souffrent souvent des limitations suivantes \citep{jain2004multibiometric} :
\begin{itemize}
	\item \textbf{Bruit dans les données acquises : }introduit par le capteur pendant l’acquisition, il peut résulter d'un capteur défectueux ou mal entretenu. Par exemple, une image d'empreinte digitale avec une cicatrice, un échantillon de voix altéré par le froid ...etc.
	\item \textbf{Variation intra-classe : }variation entre les échantillons de la même modalité biométrique d’un même individu, elle peut être causée par une mauvaise interaction de l'utilisateur avec le capteur comme les changements de pose et d'expression faciale lorsque l'utilisateur se tient devant une caméra; elle augmente généralement le taux de faux rejets (FRR) de système biométrique. 
	\item \textbf{Similarité interclasse : }les caractéristiques extraites à partir de données biométriques d'individus différents peuvent être relativement similaires. Par exemple, une certaine partie de la population peut avoir une apparence faciale similaire due à des facteurs génétiques. Cela peut augmenter le taux de fausses acceptations (FAR) du système.
	\item \textbf{Non-universalité : }certains individus de population peuvent être incapables de présenter une modalité biométrique pour le système biométrique en raison d'une maladie ou d'une incapacité. 
	\item \textbf{Sensibilité aux attaques : }implique la falsification des modalités biométrique afin d'effectuer la reconnaissance. Les modalités les plus sensibles à ce genre d'attaque sont les modalités biométriques comportementales telles que la signature et la voix.
\end{itemize}
\section{Multi-biométrie}
Pour pallier aux limites des systèmes biométriques uni-modaux qu'on a déjà présentées dans la section \ref{section:limitation}, les systèmes multi-biométriques qui combinent des informations issues de multiples sources d’information sont une solution fiable pour aborder ces problèmes. En combinant plusieurs informations discriminantes, on souhaite améliorer le pouvoir de reconnaissance du système et augmenter la robustesse aux fraudes.\\
Dans ce qui suit, nous allons étudier les systèmes multi-biométriques, en commençant par présenter les différentes formes des systèmes multi-biométriques. Nous détaillerons par la suite une de ses formes à savoir « les systèmes multimodaux » en présentant ses avantages et ses différentes architectures. Enfin, nous exposerons la fusion multimodale et ses niveaux.
\subsection{Formes des systèmes multi-biométriques}

La reconnaissance dans un système multi-biométrique est effectuée à partir de multiples sources d'informations biométriques. Selon la nature de ces sources, les systèmes multi-biométriques peuvent être divisés en six formes \citep{ross2008introduction} : multi-capteur, multi-algorithme, multi-instance, multi-échantillon, multimodaux et forme hybride (voir figure \ref{fig:chapitre1sysmultibio})
\begin{center}
	\begin{figure}[H]
		\centering
		\fbox{\includegraphics[width=0.75\linewidth]{sysmultibio}}
		\caption{Formes de systèmes multi-biométriques \citep{jain2007technology}.}
		\label{fig:chapitre1sysmultibio}
	\end{figure}
\end{center}

Le tableau \ref{tab:comparaisonFormes} présente une comparaison entre les formes de systèmes multi-biométriques selon le nombre de sources d'information utilisées.
\begin{table}[H]
	\centering
	
	
	\begin{tabular}{|p{3cm}|p{3.2cm}|p{3.2cm}|p{4cm}|p{2cm}|}
		
		% header and footer information
		\hline
		
		\centering\textbf{Système multi-biométrique }
		& \centering	\textbf{Capteur}
		& \centering \textbf{Algorithme}
		& \centering \textbf{Instance} 
		& \textbf{Modalité} \\ \hline
		Multi-capteur&Toujours 2 &Généralement 1 \textbf{*}&Toujours 1 même modalité et même instance&Toujours 1\\ \hline
		Multi-algorithme&Toujours 1 &Toujours 2 &Toujours 1 &Toujours 1 \\ \hline
		Multi-instance&Généralement 1 \textbf{**}&Toujours 1 &2 instances d'une seule modalité&Toujours 1 \\ \hline
		Multi-échantillon&Toujours 1 &Toujours 1 &2 échantillons d'un seule modalité&Toujours 1 \\ \hline
		Multimodal&Généralement 2 \textbf{***}&Plusieurs&2 toujours&Plusieurs \\ \hline
	\end{tabular}
	\caption{Comparaison entre les différents formes de la multi-biométriques selon la source d'information \citep{dhamala2012multibiometric}.}	
	\label{tab:comparaisonFormes}
\end{table} 

\textbf{Exceptions : }
\begin{itemize}
	\item \textbf{* }Il est possible que deux échantillons provenant de différents capteurs soient traités en utilisant deux différents algorithmes d'extraction de caractéristiques biométriques. Puis, un algorithme d'appariement commun.
	\item \textbf{** }Dans certains cas, on peut utiliser deux capteurs capturant chacun une instance.
	\item \textbf{*** }Un système multimodal avec un seul capteur utilisé pour capturer deux modalités différentes, par exemple une image d’une résolution élevée utilisée pour extraire le visage et l'iris.
\end{itemize}
\subsubsection{Système multi capteur }
Plusieurs capteurs sont utilisés dans l'acquisition d'une seule modalité biométrique dans le but d'acquérir des informations complémentaires pour accroître les performances des systèmes uni-modaux. Comme exemple, nous pouvons citer l’utilisation d’un capteur optique et d’un autre capacitif pour l'acquisition de l’empreinte digitale.
\subsubsection{Système multi algorithme}
Ce type correspond aux systèmes qui utilisent plusieurs algorithmes pour traiter la même image acquise d'une même modalité biométrique. A titre d'exemple l'utilisation de deux algorithmes pour la reconnaissance des empreintes digitales, le premier traite les caractéristiques texturales, le seconde traite les minuties d’une empreinte digitale. 
\subsubsection{Système multi instance}
Ce type désigne les systèmes qui capturent plusieurs unités ou instances de la même modalité biométrique (les modalités qui possèdent plusieurs instances), et avec le même capteur. Par exemple le système de reconnaissance multi-instance d’iris utilise l’iris de l’œil droit ainsi que l’iris de l’œil gauche.
\subsubsection{Système multi échantillon }
Les systèmes où un seul capteur est utilisé pour capturer plusieurs copies de la même modalité biométrique, dans différentes positions et sous différents angles, pour obtenir une représentation plus complète. Par exemple, le cas de la reconnaissance du visage, plusieurs profils du visage sont capturés, tels que le profil frontal, le profil droit et gauche, afin de prendre en compte les variations de la pose faciale.
\subsubsection{Système multimodal}
Ce type de système combine différentes modalités biométriques du même individu. Par exemple les systèmes de reconnaissance qui fusionnent entre le visage et l'iris tel que le système présenté dans \citep{rattani2009robust}, ou l'empreinte digitale et l'empreinte palmaire \citep{chin2009integrating}.
\subsubsection{Système hybride}
Le terme système \textit{hybride} est utilisé pour décrire les systèmes qui intègrent un sous-ensemble des cinq systèmes présentés précédemment \citep{chang2005evaluation}. Par exemple, le système multimodale et multi-algorithme de \citep{brunelli1995person} qui comprend deux algorithmes pour la reconnaissance du locuteur et trois pour la reconnaissance du visage.

\subsection{Les systèmes multimodaux}
La multi-modalité consiste à combiner plusieurs preuves présentées par différentes modalités
biométriques, afin d’établir l’identité d’un individu, et d’obtenir des meilleures performances de
reconnaissance que les systèmes monomodaux \citep{ross2003information}.
Pour construire un système multimodal, une variété de facteurs doit être prise en compte lors de la conception d'un système biométrique multimodal \citep{Ross2004a}, on cite :
\begin{itemize}
	\item Le choix des modalités biométriques de base, on peut s'attendre à une amélioration de la performance en utilisant des modalités physiquement non corrélées (par exemple, l'empreinte digitale et l'iris) que l'utilisation des modalités corrélées (par exemple, le mouvement de la voix et des lèvres).
	\item Le niveau de fusion des informations fournies par multiple sources biométriques.
	\item La méthodologie adoptée pour intégrer l’information.
	Le compromis entre le coût supplémentaire et l’amélioration de la performance du système.
\end{itemize}
\subsubsection{Avantages et inconvénients de la multi-modalité }
Les systèmes multimodaux présentent plusieurs avantages, que nous présentons dans ce qui suit :
\begin{itemize}
	\item 	Les systèmes biométriques multimodaux sont capables de résoudre le problème de non universalité. Dans un système multimodal si un individu ne possède pas une modalité, on peut utiliser l’autre modalité comme alternative.
	\item 	Augmenter la précision en utilisant une stratégie de fusion afin de combiner la décision de chaque sous-système et arriver à une décision finale \citep{1_trader_2017}.
	\item 	Diminuer la possibilité des attaques en compliquant la tâche de la reconnaissance d’un individu. 
\end{itemize}	
\vspace{1cm} 
Cependant, les systèmes multimodaux ont, par rapport aux systèmes monomodaux, une période de développement et une complexité plus élevées, un coût supplémentaire résultant par l’ajout de nouveaux capteurs et une taille de données plus grande. Ainsi, un temps supplémentaire est nécessaire pour acquérir et traiter plusieurs modalités pendant la phase d’enrôlement et de reconnaissance. 
\subsubsection{Architecture des systèmes multimodaux}
Les systèmes multimodaux fusionnent plusieurs systèmes monomodaux, et nécessitent donc l'acquisition et le traitement de modalités différentes qui peuvent se faire successivement ou simultanément. On parle alors d’architecture \textbf{\textit{séquentielle}}, ou d'architecture \textbf{\textit{parallèle}}.
\paragraph*{Architecture séquentielle }: dans cette architecture, l'acquisition des images est faite séquentiellement dans un ordre prédéfini, la sortie d'une modalité est généralement utilisée pour réduire le nombre des identités possibles (nombre des individus identifiés) avant d’utiliser la modalité suivante \citep{hong1998integrating} (voir figure \ref{fig:chapitre1archiseq}). Cette architecture permet de réduire le temps global de la reconnaissance par rapport à l'architecture parallèle. Car la décision finale pourrait être faite avant l'acquisition de toutes les modalités. 
	
	\begin{center}
		\begin{figure}[H]
			\centering
			\fbox{\includegraphics[width=0.55\linewidth]{archiseq}}
			\caption{Architecture séquentielle d'un système multimodale \citep{ross2006information}.}
			\label{fig:chapitre1archiseq}
		\end{figure}
	\end{center}
\paragraph*{Architecture parallèle }: dans cette architecture, on utilise toutes les modalités de base qui sont acquises en parallèle (voir figure \ref{fig:chapitre1archipar}). C'est l'architecture la plus utilisée car elle améliore la performance et le temps d’acquisition \citep{hong1998integrating}. Cependant, elle est coûteuse en terme de temps de traitement.
	\begin{center}
		\begin{figure}[H]
			\centering
			\fbox{\includegraphics[width=0.55\linewidth]{archipar}}
			\caption{Architecture parallèle d'un système multimodale \citep{ross2006information}.}
			\label{fig:chapitre1archipar}
		\end{figure}
	\end{center}

Il est également possible de concevoir une architecture hiérarchique pour combiner les avantages de deux architectures précédentes où un sous-ensemble des modalités est acquis en parallèle et un autre en série. Cependant, ce type d'architecture n'a pas reçu beaucoup d'attention de la part des chercheurs \citep{ross2006information}.
\subsection{Fusion multimodale}
\label{fusionetat}
La fusion biométrique multimodale est un sujet d'actualité, elle permet de combiner les mesures de différentes modalités biométriques, pour renforcer les points forts et réduire les points faibles de différents systèmes biométriques fusionnés. Elle peut se faire à cinq niveaux différents : niveau capteur, niveau caractéristiques, niveau score, niveau décision ou niveau rang \citep{C.SandersonandK.Paliwal.2002}. 

\subsubsection{Niveau capteur}
Les données brutes capturées à partir des différents capteurs, ou plusieurs instances d'un seul capteur sont fusionnées pour construire un vecteur de caractéristiques. La fusion à ce niveau est la moins utilisée parce qu’il est possible d'avoir des données de différentes modalités incompatibles \citep{gudavalli2012multimodal}. Les captures utilisées doivent être compatibles. Par exemple, les images de visage obtenues à partir de plusieurs caméras peuvent être combinées pour former un modèle 3D du visage \citep{noore2015fusion}.
\subsubsection{Niveau caractéristiques}
\label{fusionmethodesscore}
Lorsque les vecteurs de caractéristiques sont homogènes\footnote{\textbf{Homogénéité des caractéristiques :} c'est lorsque les caractéristiques sont extraites d'une même modalité biométrique.} (exemple : plusieurs images de différentes instances d’une empreinte digitale), le vecteur de caractéristiques résultant peut être calculé par la somme pondérée des vecteurs de caractéristiques de chaque image, et si les vecteurs sont hétérogènes (exemple : deux modalités biométriques comme le visage et la géométrie de la main), ils peuvent être concaténés pour obtenir le vecteur de caractéristiques final. Néanmoins dans le cas d'incompatibilité des vecteurs de caractéristiques initiaux, la fusion n’est pas possible \citep{ross2006information}, Par exemple, les minuties d’une empreinte digitale et les coefficients de visage ("eigen-face coefficients").
\subsubsection{Niveau score }
On fusionne à ce niveau les scores donnés en sortie de la phase d’appariement de chaque modalité biométrique pour former un score unique qui est ensuite utilisé pour prendre la décision finale. Ce niveau est le plus utilisé.
\\	\textbf{Normalisation des scores : }
les scores combinés peuvent être homogènes et donc ne nécessiteraient aucun traitement, comme ils ne peuvent pas l’être. Par exemple, dans le cas de scores non homogènes : un algorithme qui donne en sortie une mesure de distance et une autre qui donne une mesure de similarité ou des scores qui suivent des différentes distributions statistiques, alors doivent d’abord être normalisés dans un domaine commun \citep{meyer2009}. Nous présentons les méthodes de normalisation les plus connues : 
\begin{itemize}
	\item \textbf{Min-Max (MM) : }cette méthode normalise les scores bruts appartenant à l’intervalle [0, 1] \citep{ross2006information}.
	\begin{center}
		\begin{equation}\label{eq:minmax}
		n =\dfrac{(s - min (S))}{(max (S)-min (S))}
		\end{equation}
	\end{center}
$ s $ : représente un score parmi l'ensemble de scores $ S $,	max (S) et min (S) définissent les points d'extrémité du domaine de définition des scores.
	\item\textbf{Z-score (ZS) : }cette méthode transforme les scores à une distribution avec une moyenne égale 0 et un écart type égale 1 \citep{ross2006information}.
	
	\begin{center}
		\begin{equation}\label{eq:ZS}
		n =\dfrac{(s - moyenne (S)) }{(std (S))}
		\end{equation}
	\end{center}
	Moyenne (S) et std (S) désignent respectivement la moyenne et le standard de déviation des scores.
	\item \textbf{Tanh (TH) : }
	cette méthode est parmi les techniques statistiques les plus solides. Elle trace les scores de 0 à 1 \citep{ross2006information}.
	\begin{center}
		
		\begin{equation}
		\label{eq:TH}
		n = \frac{1}{2}(tanh(\dfrac{0.01*(s - moyenne (S)) }{std (S)})+1)
		\end{equation}
	\end{center}
\end{itemize}

Après la normalisation, on utilise une méthode de combinaison parmi les méthodes suivantes :
\begin{itemize}
	\item \textbf{Somme Simple :} le score final est égal à la moyenne des scores de différentes méthodes d'appariement.
	\item \textbf{Somme Pondérée :} dans ce cas, nous attribuons un poids $ w $ à chaque méthode d'appariement, le calcul de poids est basé sur leur taux de erreur ERR, et le score final est égal à la somme pondérée de ces scores.	
	\item \textbf{Min-Max :} le score final est le maximum ou le minimum score parmi l'ensemble des scores.
	\item \textbf{Médian :} le score final est le médian de l'ensemble des scores.
	\item \textbf{Produit :} le score final est le produit des scores.
\end{itemize}
\subsubsection{Niveau décision}
Les décisions des différents systèmes biométriques sont fusionnées par une stratégie de décision, parmi plusieurs stratégies existantes comme la technique de vote majoritaire, ou en utilisant les opérateurs (ET, OU, Aléatoire, etc.). Cependant, la fusion à ce niveau est la moins performante, et elle est utilisée comme une alternative lorsque les autres niveaux sont inaccessibles \citep{jain2004multibiometric}. 
\subsubsection{Niveau rang }
Ce niveau de fusion est spécifique aux systèmes multi biométriques fonctionnant dans le mode d'identification, où la sortie de chaque classificateur est un sous-ensemble de correspondances possibles, triées dans un ordre décroissant de confiance. Ces sous-ensembles sont combinés pour obtenir un rang final.	

\section{Conclusion}
Nous avons présenté dans ce chapitre des généralités sur la biométrie, notamment les types de modalités biométriques, les domaines d’application, ensuite nous avons défini les systèmes biométriques ainsi que leurs diverses domaines, leurs modes de fonctionnement. Nous avons introduit la multi-biométrie, sa définition et ses formes. Et nous avons détaillé la forme qui nous intéresse le plus qu’est « la multimodalité ». C'est la combinaison de deux modalités biométriques ou plus. Elle est utilisée afin de minimiser le taux d’erreur et de bénéficier des avantages des modalités qui la composent.
\\Parmi les systèmes multimodaux, il existe des systèmes qui fusionnent l’empreinte digitale et l’empreinte palmaire qui sont les modalités biométriques les plus connues, les plus utilisées et les plus compatibles. Les processus de reconnaissances de ses modalités seront présentés en détails dans les chapitres qui suivent.



\lhead{SYSTEME BIOMETRIQUE BASE SUR L'EMPREINTE DIGITALE}
\chapter{Système biométrique basé sur l'empreinte digitale}
\label{Chapter2} % Change X to a consecutive number; for referencing this chapter elsewhere, use \ref{ChapterX}

\section{Introduction}
\tab Parmi les nombreuses modalités biométriques existantes, l'empreinte digitale (appelée aussi « dermatoglyphe ») est la plus utilisée pour la reconnaissance des personnes grâce à son unicité, son universalité, aisance de son acquisition et sa permanence \citep{maltoni2009handbook}. La figure \ref{chapitre2fingerstat} ci-dessous présente une statistique faite en \textit{2016} sur l'utilisation des modalités dans les systèmes biométriques.

\begin{figure}[H]
	\centering
	\fbox{\includegraphics[width=0.7\linewidth]{fingerstat}}
	\caption{Statistiques sur l'utilisation des modalités dans les systèmes biométriques\citep{Counter2016}.}
	\label{chapitre2fingerstat}
\end{figure}
Dans ce chapitre, nous présentons les caractéristiques de l’empreinte digitale. Nous détaillons les différentes étapes de son processus de reconnaissance et nous présentons une variété de méthodes utilisées dans chaque étape.

\section{Caractéristiques des empreintes digitales}

L'empreinte digitale se compose de motifs dessinés par les crêtes et les vallées de la peau. Les caractéristiques liées à l'empreinte digitale sont généralement catégorisées en trois niveaux \citep{hasan2013fingerprint} :
\begin{enumerate}
	
	
	\item 	\textbf{Les détails de niveau 1: }les caractéristiques globales (les points singuliers) visibles à l'œil. Il en existe deux types : les points cores \footnote{\textbf{Un core :} est le lieu de courbure maximale des lignes d'empreinte les plus internes.}, et les deltas \footnote{\textbf{Un delta :} est le lieu de divergence des lignes les plus internes, en d'autres termes un delta est proche du lieu où se séparent deux lignes.}, et sont considérés comme le centre de l'empreinte digitale, ces points sont utilisés dans l'alignement des empreintes digitales lors de la phase d'appariement (voir figure \ref{fig:chapitre2fingerprintlevel1}). 
	
	
	\begin{center}
		\begin{figure}[H]
			\centering
			\includegraphics[width=0.20\linewidth]{fingerlevel1}
			\captionsetup{justification=centering}
			\caption{Les points singuliers d'empreintes digitales.}
			\label{fig:chapitre2fingerprintlevel1}
		\end{figure}
	\end{center}
	
	
	\item \textbf{Les détails de niveau 2 : }sont les caractéristiques locales (les minuties), ils peuvent être les fins de lignes (terminaisons), les bifurcations, les lacs, les ponts et les iles. Elles sont utilisées par la plupart des systèmes automatisés de reconnaissance et peuvent être extraites de manière fiable à partir des images d'empreintes digitales avec une faible résolution (500 dpi), cette résolution est la résolution standard adoptée par le Bureau Fédéral d'Investigation dans leurs systèmes automatiques d'identification (AFIS) \citep{jain2007pores}.
	
	
	\item \textbf{Les détails de niveau 3 :} sont la forme des bords de crêtes, les crêtes immatures et le contour des arêtes, etc. (voir figure \ref{fig:chapitre2fingerprintlevel3}). Ce niveau est peu utilisé par les systèmes de reconnaissance, car les images capturées pour extraire les détails de ce niveau sont de haute résolution (plus de 1 000 dpi). 
	
	\begin{center}
		\begin{figure}[H]
			\centering
			\includegraphics[width=0.56\linewidth]{fingerlevel3}
			\caption{Les caractéristiques d'empreinte digitale du niveau 3.}
			\label{fig:chapitre2fingerprintlevel3}
		\end{figure}
	\end{center}
\end{enumerate}
\clearpage
Le processus de la reconnaissance des empreintes digitales passe par plusieurs phases, la première phase est l'acquisition d'images d'empreintes digitales, ensuite un prétraitement des images est fait afin d'améliorer leur qualité, suivi par une extraction de données utiles et enfin, après une comparaison entre le modèle en entrée au moment de la reconnaissance et le modèle déjà enrôlé une décision est prise. Pour réduire le nombre de comparaisons d'une empreinte digitale avec les empreintes digitales stockées dans une grande base de données, les images peuvent être classifiées en procédant par une méthode de classification.
\\
Il existe trois types d'approches des systèmes de reconnaissance : approches basées sur la corrélation, approches basées sur les minuties et approches basées sur les textures. 
Les approches basées sur les minuties sont les plus utilisées par les systèmes biométriques \citep{jiang2000fingerprint} car elles donnent des résultats plus précis \citep{o1998overview}, et ce sont les approches auxquelles nous nous intéressons le plus dans notre travail. 

\section{Extraction des minuties}
Une fois toutes les étapes de pré-traitement (voir annexe \ref{pretrait}) sont appliquées et l'image binarisée et squelettisée de l'empreinte digitale est obtenue, nous extrayons les minuties \citep{tisse2001systeme}. Il y a deux types de méthodes d'extraction des minuties : 
\begin{itemize}
	\item Les méthodes basées sur la binarisation (avec ou sans squelettisation).
	\item Les méthodes directes (sans passer par le prétraitement).
\end{itemize}
La majorité des systèmes de reconnaissance utilisent les méthodes basées sur la binarisation, avec une étape de squelettisation pour avoir une image de l'empreinte de qualité meilleure \citep{bansal2011minutiae}. Dans certaines méthodes d'extraction, nous procédons par une étape d'élimination des fausses minuties qu'on présente dans l'annexe \ref{posttrait}.
\subsection{Extraction avec le nombre de connexions (CN)}
\label{chp2cn}
La plupart des recherches sur la reconnaissance des empreintes digitales extraient les minuties à l'aide de la méthode du nombre de connexions CN (Crossing Number) \citep{Thai2003}.
\\
Le nombre CN d'un pixel P dans une image binarisée est calculé comme suit \citep{maltoni2009handbook} :
\begin{center}
	\begin{equation}\label{eq:cn}
	CN (P) = \frac{1}{2}\sum_{i=1}^{8}|val (P_{i mod 8 } )- val(P_{i-1}) |
	\end{equation}
\end{center}
$ P_{i} $ est la valeur du pixel voisin à celui pour lequel le CN est calculé.
\\En utilisant les propriétés du CN, chaque pixel d'une crête peut être classé comme un point intermédiaire (non-minutie), une terminaison ou une bifurcation.

\begin{center}
	\begin{figure}[H]
		\centering
		\includegraphics[width=0.6\linewidth]{cn}
		\caption{Nombre de connexions ; a). Point intermédiaire b). Terminaison ; c). Bifurcation \citep{maltoni2009handbook}.}
		\label{fig:chapitre2cn}
	\end{figure}
\end{center}

\subsection{Extraction à base morphologique}
Ces techniques d'extraction sont basées sur une morphologie mathématique dans laquelle l'image est pré-traitée, ensuite, un autre traitement avec des opérateurs morphologiques (Ouverture) et (Fermeture) est effectué, où l'opération d'ouverture est pour la suppression des pics introduits par le bruit de fond et l'opération de fermeture pour l'élimination de petites cavités, et enfin, les minuties sont extraites à l'aide de la transformation morphologique en tout ou rien « hit or miss transformation », la technique développe une structure pour chaque type de minutie, par exemple les terminaisons de crête sont les pixels d'une image qui n'ont qu'un voisin dans un voisinage de $ 3 * 3 $. Des exemples d'utilisation de cette technique dans \citep{humbe2007mathematical} et \citep{bansal2010effective}.


\section{Appariement}
L'appariement des empreintes digitales est une étape cruciale dans les problèmes d'authentification et d'identification, elle consiste à comparer entre deux empreintes digitales et retourne un degré de similarité (un nombre réel appartient à un intervalle) ou décision binaire (accepté ou non-accepté). Il existe deux grandes approches pour l'appariement des minuties : une approche globale et une approche locale \citep{maltoni2009handbook}.

\subsection{Formulation du problème d'appariement}
Nous représentons dans ce qui suit le modèle enrôlé d'une empreinte digitale par $ (T) $ et le modèle en entrée par $ (E) $, et chaque élément du vecteur de caractéristiques (vecteur de minuties) en sortie par $ (M) $, un élément du vecteur (minutie) est désigné par : 
\begin{itemize}
	\item son type (terminaison ou bifurcation);
	\item ses coordonnées cartésiennes $ (x, y) $;
	\item son orientation ($\theta$).
\end{itemize}
	$ (T) $ et $ (E) $ sont représentés par les équations \ref{eq:t} et \ref{eq:e} respectivement : 
\begin{center}
	\begin{equation}\label{eq:t}
	\abovedisplayskip
	\belowdisplayskip
	T=(M_{1},M_{2}, ..., M_{n})\qquad M_{i}=(x_{i},y_{i},\theta_{i})\quad i = 1 .. n
	\end{equation}
\end{center}

\begin{center}
	\begin{equation}\label{eq:e}
	E=(M\prime_{1},M\prime_{2}, ..., M\prime_{n\prime})\qquad M\prime _{i\prime }=(x\prime _{j },y\prime_{j},\theta\prime_{j}) \quad j = 1 .. n\prime
	\end{equation}
\end{center}
\clearpage
La représentation graphique d'une minutie est illustrée dans la figure \ref{fig:chapitre2minutiarepresentation}.
\begin{center}
	\begin{figure}[H]
		\centering
		\includegraphics[width=1\linewidth]{minutiarepresentation}
		\captionsetup{justification=centering}
		\caption{Représentation basique des minuties a) une terminaison, b) une bifurcation \citep{maltoni2009handbook}.}
		\label{fig:chapitre2minutiarepresentation}
	\end{figure}
\end{center}
Une minutie $ M_{j} $ dans $ T $ et une minutie $ M_{i} $ dans $ E $ sont considérées appariées, si $ M_{j} $ tombe dans la zone de tolérance de $ M_{i} $. Une zone de tolérance est définie par une distance spatiale ($ sd $) maximale et une différence directionnelle ($ dd $) afin de compenser les erreurs inévitables faites lors de la phase d'extraction des minuties ou par les changements de positionnement produits par des distorsions dans l'empreinte digitale (voir les équations \ref{eq:sd} et \ref{eq:dd}). Pour maximiser le nombre de minuties de $ E $ qui correspondent à $ T $ le modèle enrôlé, il nous faut un alignement de deux modèles (cela inclut également le déplacement, la rotation et d'autres transformations géométriques), après cet alignement, un score de similarité entre les deux modèles est calculé.

\begin{center}
	\begin{equation}	
	\label{eq:sd}	
	sd(M_{i},M\prime_{j})=\sqrt{(x_{i}-x\prime _{j})^{2}+(y_{i}-y\prime _{j})^{2}} \; \leq r_{0}
	\end{equation}
\end{center}
\begin{center}
	\begin{equation}\label{eq:dd}	
	dd(M_{i},M'_{j})=\min(\mid\theta_{i}-\theta \prime _{j}\mid, 360^{o} - \mid\theta_{i}-\theta \prime _{j}\mid) \; \leq \theta_{0}
	\end{equation}
\end{center}
Le score de similarité est souvent calculé par l'équation suivante \citep{maltoni2009handbook} :
\begin{center}
	\begin{equation}\label{eq:ss}	
	Score\;de\; similarite=\dfrac{k}{\dfrac{n+n \prime}{2}}
	\end{equation}
\end{center}
\textbf{$ k : $} représente le nombre de minuties appariées, $ n $ le nombre de minuties dans $ T $ et $ n \prime $ le nombre de minuties dans $ E $. 

\subsection{Approches globales }
L'alignement dans ces approches est une étape obligatoire afin de maximiser le nombre de minuties appariées, il est exécuté en prenant en considération toutes les minuties dans leur ensemble global, par le calcul de paramètres de transformation : déplacement (en $ x $ et $ y $), rotation ($ \theta $) et d'autres informations comme le changement d'échelle (dans le cas où les deux empreintes sont acquises par des capteurs de résolutions différentes). \\
La figure \ref{fig:alignement} représente les étapes de l'appariement global : 
\begin{center}
	\begin{figure}[H]
		\centering
		\includegraphics[width=0.55\linewidth]{alignement}
		\caption{Processus générique d'un appariement global \citep{Jianjiang2010Finger}.}
		\label{fig:alignement}
	\end{figure}
\end{center}

Après l'enrôlement de chaque modèle, certaines approches globales passent par une phase de pré-alignement qui est basée sur d'autres caractéristiques extraites telles que les points singuliers ou la carte d'orientation. Ces caractéristiques seront sauvegardées dans la base de données avec le modèle. Dans le suivant, nous présentons quelque approches globales. 
\subsubsection{Approche basée sur la transformée de Hough}
\label{chp2hough}
\citep{ratha1996real} ont proposé une approche d'appariement basée sur la transformée généralisée de Hough (voir annexe \ref{Hough}), dont la transformation d'alignement est estimée en discrétisant l'espace de recherche. Selon Mr. Maltoni, cette approche est la plus représentative de l'appariement global \citep{maltoni2009handbook}.

\subsubsection{Approche basée sur la géométrie algébrique}
C'est une approche plus simple que l'approche précédente \citep{maltoni2009handbook}, elle était introduite par \citep{udupa2001fast} qui ont considérablement amélioré une idée précédemment publiée par \citep{weber1992cost}, les changements d'échelle qui sont des transformations rigides ne sont pas autorisés dans cette approche. 
\subsubsection{Approche basée sur le pré-alignement absolu}
Un pré-alignement est opéré sur chaque image de la base de données indépendamment d'autres images avant d'être stockées. La méthode de M82 du FBI est la méthode la plus populaire de cette approche, elle effectue un pré-alignement absolu en fonction de la position du core détectée par la méthode R92 \footnote{Rule-based (R92) : une méthode basée sur les règles utilisée pour détecter le core de l’empreinte digitale en exploitant la carte d’orientation.}.

\subsubsection{Approche basée sur le pré-alignement relatif}
Le pré-alignement sur chaque image dépend des autres images de la base de données. Cette approche est plus robuste que l'approche absolue, mais moins rapide, le pré-alignement peut être effectué de plusieurs manières :
\begin{itemize}
	\item en superposant les points singuliers après la détection de la position du point singulier central (core ou delta);
	\item en corrélant les images d'orientation par le calcul de la similarité entre la carte d'orientation (voir annexe \ref{carteOM}) de chaque image avec toute transformation possible des cartes d'orientation des autres images;
	\item en comparant les caractéristiques des crêtes (par exemple, la longueur et l'orientation des crêtes).
\end{itemize}
\subsubsection{Autres approches globales d'appariement}
Il existent d'autres approches basées sur l'appariement global des minuties dans la littérature comme l'approche basée sur la modélisation de Warping \citep{meenen2006utilization}, \citep{liang2006fingerprint} et \citep{shi2009minucode}, et l'approche basée sur les algorithmes évolutifs \citep{sheng2009consensus}, \citep{sheng2007memetic} et \citep{tan2006fingerprint}, et d'autres approches expliquées dans le livre « \textit{Guide de la reconnaissance d'empreintes digitales} \citep{maltoni2009handbook}».
\subsection{Approches locales }
Malgré que l'approche globale conduit vers une distinctivité plus élevée car elle prend en considération les relations spatiales entre les minuties sur le plan global, les approches locales sont plus simples et possèdent une faible complexité de calcul et une tolérance à la distorsion vu qu'elles nous permettent d'apparier deux minuties même avec des informations partielles, car elles consistent à comparer deux empreintes digitales selon les structures locales des minuties. Ces structures se caractérisent par des propriétés invariantes par rapport à la transformation globale, telles que les déplacements et les rotations.
\\ Pour obtenir les avantages des deux types d'approches, on utilise des stratégies \textit{\textbf{hybrides}} qui effectuent un appariement des structures locales suivi d'une étape de consolidation. La première étape détermine les paires de minuties qui correspondent localement, et extrait un sous-ensemble d'alignements candidats pour $ (T) $ et $ (E) $, et la deuxième étape, vise à vérifier dans quelle mesure les correspondances locales détiennent au niveau global. 

Les techniques d'appariement locales qui existent dans la littérature se différencient entre elles dans la topologie de la structure locale utilisée, le type de consolidation, l'utilisation des caractéristiques supplémentaires, l'utilisation de particularités des minuties et l'apprentissage des paramètres \citep{Peralta2015a} :
\subsubsection{Topologie de la structure locale }
\label{structures}
L'appariement local est basé sur le calcul de la similarité entre les régions locales de $ T $ et $ E $, dans le but d'obtenir l'invariance souhaitée en ce qui concerne les déplacements et les rotations. Ces régions sont associées à des sous-ensembles de minuties qui sont classées en structures locales et ils peuvent être construits sous différentes manières dont :
\begin{itemize}
	\item \textbf{Les plus proches voisins :} la structure locale est formée par une minutie centrale et un certain nombre prédéterminé de ses plus proches minuties, ce type de structure est généralement défini par une distance, une direction et un angle entre les paires de minuties \citep{jiang2000fingerprint}.
	\item \textbf{Le rayon fixe :} la structure locale est créée à partir d'une minutie centrale et ses voisins dans une courbe de rayon R \citep{ratha2000robust}.
	\item \textbf{La texture mixte :} la structure locale est définie comme un vecteur de caractéristiques qui contient des informations appropriées extraites de la minutie et d'autres types d'informations provenant d'autres caractéristiques extraites de l'image d'empreinte, telles que l'orientation locale, la fréquence des crêtes, la représentation d'image en niveau de gris ou autres \citep{benhammadi2007fingerprint}.
	\item \textbf{Les triplets des minuties : }cette méthode de structuration est appelée aussi les triangles des minuties, les triplets sont construits sous forme de triangulation, en suite pour former les structures locales on utilise des informations extraites à partir de ces triangles comme les angles des sommets, la longueur des côtés et certaines propriétés du triangle telles que la direction et l'orientation \citep{maltoni2009handbook}.
	\item \textbf{K-Plet :} c'est une dérivation de la structure locale « plus proches voisins » présentée par \citep{chikkerur2006k}, où les voisins les plus proches sont également répartis dans les quatre quadrants autour de la minutie centrale.
	\item \textbf{Le cylindre de minutie :} une extension de la structure locale à rayon fixe, elle permet d'encoder les relations spatiales et directionnelles pour chaque minutie afin de rendre le calcul de la similarité des structures locales plus simple \citep{cappelli2010minutia}.
	\\
	La figure \ref{fig:chapitre2localTypes} illustre les différentes structures locales précédemment présentées.
	
	\begin{figure}[H]
		\centering
		\includegraphics[width=0.9\linewidth]{localTypes}
		\caption{Types des structures locales.}
		\label{fig:chapitre2localTypes}
	\end{figure}
\end{itemize}	
\subsubsection{Consolidation}
Bien que, à partir des scores partiels de la comparaison des structures locales nous pouvons avoir un score final, mais généralement, on procède par une phase supplémentaire pour vérifier si l'appariement local de deux minuties correspond au niveau global \citep{maltoni2009handbook}. Parmi les différentes techniques de consolidation qui ont été proposées :
\begin{itemize}
	\item \textbf{La transformation unique :} basée sur l'alignement des minuties centrales de T et E en utilisant la meilleure transformation (le déplacement et la rotation) provenue de minuties qui possèdent le meilleur score d'appariement local \citep{jiang2000fingerprint}.
	\item \textbf{La transformation multiple :} plusieurs transformations sont effectuées dans l'alignement des deux structures locales, ce type de transformation est employé dans le cas d'une empreinte digitale de mauvaise qualité ou une empreinte déformée \citep{maltoni2009handbook}.
	\item \textbf{Le consensus des transformations : }consiste à trouver le nombre maximal de transformations cohérentes pour chaque structure locale \citep{maltoni2009handbook}.
	\item \textbf{La consolidation incrémentale :} ce type est compatible seulement avec les structures locales arrangées sous forme des graphes orientés où les K-plets sont les nœuds. L'appariement est effectué en parcourant les graphes en largeur et enfin le nombre des nœuds appariés est retourné, ce processus est répété pour chaque paire de minutie et la paire ayant le meilleur score est choisie \citep{chikkerur2006k}.
\end{itemize}
\subsubsection{Utilisation des caractéristiques supplémentaires}
Dans certaines méthodes d'appariement, il est possible d'utiliser des caractéristiques supplémentaires recueillies auprès d'autres sources comme les informations extraites à partir de l'image d'orientation locale ou l'estimation locale de fréquence de crêtes \citep{Peralta2015a}. Les caractéristiques supplémentaires qui peuvent être utilisées sont les suivantes :
\begin{itemize}
	\item \textbf{La fréquence des crêtes : }représente la distance moyenne locale entre les crêtes sur un bloc et peut être utilisée comme une caractéristique locale associée aux minuties, quand elle est relativisée par rapport à la fréquence des crêtes globales ou pour normaliser les distances entre deux minuties \citep{chikkerur2007fingerprint}.
	\item \textbf{Les points singuliers :} les positions et les orientations des points singuliers peuvent être utilisées dans l'appariement local, comme notamment dans l'article \citep{zhang2002core} et \citep{feng2008combining}.
	\item \textbf{L'orientation locale des crêtes :} l'image est divisée en blocs qui ne se chevauchent pas et une valeur d'orientation est calculée à partir l'orientation de chaque pixel composant le bloc. La valeur d'orientation du bloc peut être associée à la minutie centrale d'une structure locale \citep{maltoni2009handbook}. 
	\item \textbf{L'image en niveau de gris :} elle inclut des informations sur la texture telles que les régions d'image d'empreinte améliorée par des filtres \citep{Peralta2015a}.
\end{itemize}
\subsubsection{Particularités dans les minuties}
Ce sont les informations complémentaires étroitement liées à la minutie. Les plus importantes sont :
\begin{itemize}
	\item \textbf{Types de minutie :} il existe plusieurs types. Les plus utilisés sont les \textit{\textbf{bifurcations}} et les \textit{\textbf{terminaisons}}. Car les autres types de minuties ne sont que des résultats de combinaisons de ceux-ci. Par exemple, les iles peuvent être visualisées en tant que deux bifurcations (voir Figure \ref{fig:chapitre2types}). 
	\begin{center}
		\begin{figure}[H]
			\centering
			\fbox{\includegraphics[width=0.55\linewidth]{chapitre2types}}
			\caption{Différents types de minuties : (a)terminaison (b) bifurcation, (c) pont, (d) lac et (e) ile.}
			\label{fig:chapitre2types}
		\end{figure}
	\end{center}
	\item\textbf{Nombre de crêtes :} représente le nombre de crêtes associées à chaque minutie centrale de la structure locale.
	\item \textbf{Propriétés des crêtes :} comme le rayon de courbure de crête à laquelle la minutie appartient.
\end{itemize}
\subsubsection{Apprentissage de paramètres }
Des techniques d'apprentissage qui se basent sur les machines learning sont généralement employées dans l'optimisation du score de similarité qui détermine la décision finale. Les formes d'apprentissage des paramètres sont les suivantes :
\begin{itemize}
	\item \textbf{Score de similarité :} une fonction qui reçoit les vecteurs de caractéristique représentant deux structures locales de $ E $ et $ T $ et donne comme résultat le score de similarité optimisé qui est appris à l'aide des réseaux de neurones ou d'autres schémas de régression.
	\item \textbf{Similitude locale :} un processus d'apprentissage hors ligne est effectué pour apprendre la vraie similitude entre les structures locales ou pour ajuster les poids de contribution associés à chaque élément du vecteur de caractéristiques.	
\end{itemize}
\subsubsection{Synthèse des travaux sur l'appariement local basé sur les minuties}
Dans la littérature, plus de 80 méthodes d'appariement local basées sur des minuties ont été proposées \citep{Peralta2015a}.
Le tableau \ref{tab:chapitre2fingermatching} résume quelques travaux de recherche.

\begin{sidewaystable}[h!]
	\centering
	
	\begin{tabular}{|p{4cm}|p{4cm}|p{4cm}|p{3cm}|p{3cm}|p{4cm}|}
		\hline
		\begin{center}
			\textbf{Topologie de la structure locale}
		\end{center} &\begin{center}
			\textbf{Le type de consolidation}
		\end{center} &\begin{center}
			\textbf{Caractéristiques supplémentaires utilisés}
		\end{center} & \begin{center}
			\textbf{Les particularités dans les minuties} 
		\end{center}& \begin{center}
			\textbf{La forme d'apprentissage des paramètres}
		\end{center} &\begin{center}
			\textbf{La référence}\begin{center}
				
			\end{center}
		\end{center} \\ \hline
		La texture mixte & Transformation unique & 
		L'orientation locale des crêtes
		& Types de minutie et Propriétés des crêtes & Aucune & \citep{he2003image} \\ \hline
		K-Plet & Incrémentale & Les points singuliers & Types de minutie & Aucune & \citep{chikkerur2005impact} \\ \hline
		La texture mixte et les triplets des minuties & Le consensus des transformations & L'orientation locale des crêtes & Aucune & Similitude locale & \citep{chen2006algorithm} \\ \hline
		La texture mixte & Transformation multiple & L'image en niveau de gris & Aucune & Aucune & \citep{benhammadi2007fingerprint} \\ \hline
		Les plus proches voisins & Incrémentale & Aucune & Aucune & Aucune & \citep{Watson2010} \\ \hline
		Le rayon fixe et la texture mixte & Transformation multiple & La fréquence des crêtes et l'orientation locale des crêtes & Propriétés des crêtes & Score de similarité & \citep{cao2009fingerprint} \\ \hline
		Les plus proches voisins & Transformation unique & Aucune & Les types de minutie et le nombre de crêtes & Aucune & \citep{jiang2000fingerprint} et \citep{bengueddoudj2013improving} \\ \hline
		La texture mixte et les triplets des minuties & Aucun & L'image en niveau de gris et les points singuliers & Aucune & Aucune & \citep{mistry2013fusion} \\ \hline
	\end{tabular}
	\caption{Quelques travaux de recherche sur l'appariement local basé sur les minuties.}% Add 'table' caption	
	\label{tab:chapitre2fingermatching}	
\end{sidewaystable}
\clearpage
\section{Classification des empreintes digitales}
\label{fingerprintclassification}
La classification des empreintes digitales est une technique efficace qui permet de réduire le nombre de comparaisons d'une empreinte digitale avec les empreintes digitales stockées dans une grande base de données, par conséquent cela va permettre de réduire le temps de recherche.
\\ Le principe est de partitionner la base de données en plusieurs classes en utilisant des caractéristiques extraites de l'empreinte digitale (par exemple : le nombre et la position de points singuliers, les orientations, les réponses aux filtres de Gabor, etc.), ensuite attribuer chaque empreinte digitale enrôlée à une classe. \\
Les approches de classification existantes peuvent être attribuées à l'une des catégories suivantes : les approches syntaxiques, les approches structurales, les approches statistiques, les approches neuronales, les approches qui utilisent les SVM. Ces approches sont soit des approches fixes ou basées sur des techniques d'apprentissage, nous les présentons dans ce qui suit :
\subsection{Système de classification de Henry}
Les premières recherches scientifiques sur la classification des empreintes digitales ont été faites par \citep{galton1892finger}, qui a divisé les empreintes digitales en trois grandes classes. Plus tard, Henry et Edward Richard ont redéfini la classification de Galton en augmentant le nombre des classes à cinq \citep{henry1905classification} : la boucle droite (Right Loop (R)), la boucle gauche (Left Loop (L)), la volute (Whorl (W)), l'arche (Arch (A)) et l'arche lentiforme (Tented Arch (T))(Edward Richard, 1900), ces classes d'empreintes digitales sont inégalement réparties dans la population (3,7\%, 2,9\%, 31,7\%, 33,8\% et 27,9\%, respectivement \citep{peralta2017robust}). 
\\ Un exemple pour chaque classe sont présentés dans la figure \ref{fig:chapitre2henryclasses}. 
\begin{figure}[H]
	\centering
	\includegraphics[width=0.7\linewidth]{chapitre2henryclasses}
	\caption{Classes d'empreinte digitale : a) boucle, b) boucle droite, c) volute, d) arche, e) arche lentiforme.}
	\label{fig:chapitre2henryclasses}
\end{figure}


\subsection{Approches syntaxiques}
Ces approches sont basées sur l'extraction des symboles à partir des caractéristiques de l'empreinte. L'idée de base consiste à définir une grammaire pour chaque classe d'empreinte digitale, ensuite la classification est effectuée par une analyse syntaxique afin de déterminer quelle grammaire génère les symboles extraits\citep{mridula2014review}. La figure \ref{fig:chapitre2classificationsyntax} illustre une méthode introduite par Rao et Balck 1980 qui se base sur l'analyse des flux de lignes de crêtes \citep{karu1996fingerprint}.


\begin{center}
	\begin{figure}[H]
		\centering
		\fbox{\includegraphics[width=0.55\linewidth]{chapitre2classificationsyntax}}
		\caption{Un schéma de la méthode de Rao et Balck \citep{karu1996fingerprint}.}
		\label{fig:chapitre2classificationsyntax}
	\end{figure}
\end{center}
\subsection{Approches structurales}
Ce sont les approches qui se basent sur l'organisation relationnelle des caractéristiques qui est représentée par des structures de données symboliques, telles que les arbres et les graphes, qui permettent d'avoir une organisation hiérarchique de l'information \citep{maio1996structural}. Exemples de ces organisations : les arbres de décision ($ DT $ Decision Trees) et les modèles de Markov cachés ($ HMM $ Hidden Markov Model).
\subsection{Approches statistiques}
\label{KNN}
Dans cette approche, nous extrayons un vecteur de caractéristiques numérique de taille fixe à partir d'une empreinte digitale en se basant sur le champ d'orientation ou sur la réponse aux filtres de Gabor, nous utilisons un classificateur statistique pour la classification, parmi les classificateurs les plus utilisés, nous avons le plus proche voisin (($ k-NN $) $ k $- nearest neighbor) qui est une méthode d'apprentissage supervisée où une nouvelle empreinte digitale est classifiée à la base d'un vecteur de caractéristiques construit à partir des scores attribués, pour trouver le plus proche voisin, la distance de hamming ou angulaire est calculée entre la nouvelle empreinte digitale et chaque empreinte digitale existante dans la base de test. Le résultat final est obtenu en effectuant un tri ascendant \citep{kong2009survey}.
\subsection{Approches des réseaux de neurones }
\label{NN}
Un réseau de neurones représente un modèle de calcul composé d'entités interconnectées où l'entité est un neurone, ou une succession de couches dont chacune prend ses entrées sur les sorties de la précédente. Un neurone est caractérisé par un état d'excitation qui dépend de celui des neurones situés à la couche supérieure ainsi que de la force des liens qui les relient. Dans la majorité des cas, les neurones sont en fait des fonctions calculées par un programme informatique, mais ils sont parfois réalisés sur des circuits électroniques.
\subsection{Approche des machines à vecteurs de support (SVM) }
\label{SVM}
Cette approche englobe les méthodes d'apprentissage supervisées destinées à résoudre des problèmes de discrimination et de régression \citep{honeine2007methodes}, elles reposent sur deux principes :
\begin{itemize}
	\item \textbf{Marge maximale : }on cherche à maximiser la marge entre l'hyperplan séparateur recherché et les éléments de chaque classe de l'ensemble d'apprentissage.
	\item \textbf{Fonction noyau :} qui permet de conférer un caractère non-linéaire à nombre de traitements originellement linéaires sans qu'il soit nécessaire de recourir à d'importants développements théoriques.
\end{itemize}
Le but du SVM est de trouver une séparatrice qui minimise l'erreur de classification sur l'ensemble d'apprentissage qui sera également performante en généralisation sur des données non utilisées en apprentissage. Pour cela le concept utilisé est celui de marge (d'où le nom de séparateurs à vaste marge). La marge est la distance quadratique moyenne entre la séparatrice et les éléments d'apprentissage les plus proches de celle-ci appelés vecteurs de support. Ces éléments sont appelés vecteurs de support car c'est uniquement sur ces éléments de l'ensemble d'apprentissage qu'est optimisée la séparatrice \citep{belahcene2012comparaison}.



\section{Conclusion}

Dans ce chapitre, nous avons introduit la modalité d'empreinte digitale qui est considérée comme la modalité biométrique la plus utilisée, ensuite, nous avons présenté les trois niveaux de ses caractéristiques. Nous avons aussi expliqué le processus de reconnaissance en mettant l'accent sur l'approche basée sur les minuties. Enfin, nous avons présenté les différentes méthodes d'extraction et d'appariement des minuties. 
\\Dans le chapitre suivant nous allons présenter le processus de la reconnaissance d'empreinte palmaire.

\chapter*{Conclusion générale et perspectives}
\tab La biométrie multimodale est considérée comme la technique de reconnaissance des individus la plus sure grâce à son efficacité, performance et résistance aux attaques. A partir de ce constat, le domaine de recherche en biométrie est très actif et riche en termes de méthodes de reconnaissance biométrique de toute modalité. Cette richesse a impliqué la nécessité d’un outil de test qui permet aux chercheurs en biométrie multimodale d’évaluer leurs systèmes biométriques unimodaux et multimodaux. \\ \tab
Ce projet vient pour répondre à la problématique posée en offrant « Jupiter » une plateforme de test de systèmes biométriques unimodaux et la fusion multimodale, cette plateforme en premier lieu permet le test des méthodes des deux modalités d’empreinte digitale et d’empreinte palmaire écrite en Matlab, avec la possibilité d’étendre la plateforme en d’autres modalités et d’autres langages. \\ \tab
A travers notre plateforme, le chercheur en biométrie trouve plusieurs méthodes et bases de données de tests d’empreinte digitale et d’empreinte palmaires prêtes, il peut en ajouter des autres, tester un système biométrique de reconnaissance d’empreinte digitale, d’empreinte palmaire, de plus, tester la fusion multimodale au niveau des caractéristiques ou bien au niveau score, ainsi, il peut composer plus de 60 systèmes biométriques unimodaux et multimodaux prêts à tester. \\ \tab
« Jupiter » avec ses fonctionnalités, son extensibilité et le fait d’être un outil open source, constitue une solution parfaite pour la communauté des chercheurs en biométrie pour partager leurs tests, s’ils le veulent, donc « Jupiter » sera un espace de tests biométriques et toutes les informations concernant un test : le chercheur qui a lancé le est, les bases de données de test utilisées, la date, les résultats et les informations concernant l’environnement du test, un chercheur peut donc éviter de retester un système biométrique s’il trouve le test qu’il veut, ou bien le tester avec d’autres bases de données de tests ou effectuer le test sous un autre environnement et comparer ses résultats avec d’autres résultats de tests. \\ \tab
Afin de construire « Jupiter » telle qu’elle est maintenant, des choix ont été prises tout au long de notre travail :
\begin{enumerate}
\item Le choix des modalités à combiner : qui sont l’empreinte digitale et l’empreinte palmaire, effectué après une étude sur la biométrie et comparaison entre les modalités, nous avons sélectionné ses deux modalités à cause de leurs efficacité, performance, acceptabilité par les individus et leur universalité.

	\item Le choix des méthodes d’empreintes digitales et d’empreintes palmaires à implémenter : était basé sur l’étude faite pendant la phase d'état de l’art.
\item Les choix technologiques : pour l’architecture et les langages de programmation, sélectionnés d’une manière à garantir l’extensibilité de la solution, la réutilisabilité du code  et la réduction du temps de réponse vu le volume des BDDs de test pour renvoyer les résultats du test au chercheur rapidement.
\end{enumerate}
Aussi, pour comprendre le besoin des chercheurs en biométrie multimodale et d’y répondre, notre travail s’articulait autour de axes :
\begin{itemize}
	\item La compréhension des processus de reconnaissance d’empreinte digitale, d’empreinte palmaire et la fusion multimodale.
	\item Concevoir et réaliser une plateforme de test facilement manipulable par les chercheurs, qui concerne principalement les empreinte digitales et palmaires en garantissant l’extensibilité de la solution.
\end{itemize}
Malgré que « Jupiter » réponde aux objectifs qui lui ont été fixé, elle peut être amélioré, c’est pour cela, nous proposons les perspectives suivantes :
\begin{itemize}
	\item Ajouter d’autres modalités comme le visage, l’iris, …etc.
	\item Étendre « Jupiter » pour qu’elle supporte d’autres représentations d’empreinte digitale (représentation en texture) et d’empreinte palmaire (représentation en minuties) aussi que les images de hautes résolutions.

	\item Enrichir la plateforme avec d’autres méthodes de reconnaissance et d’autres bases de données de tests.
	\item Développer le test de fusion multimodale en ajoutant les autres niveaux de fusion : niveau capteurs et niveau décision.
    \item Permettre aux chercheurs d’écrire les méthodes en d’autres langages que Matlab, comme python, C++, …etc.
    \item Ajouter la possibilité de communication entre les chercheurs sur la plateforme et de créer des groupes de partage (laboratoire, équipe d’un projet, binôme ..etc).

    \item Donner la possibilité aux chercheurs de tester les méthodes de classification des bases de données biométriques.
\end{itemize}













\cleardoublepage
\addcontentsline{toc}{chapter}{\bibname}

%récupérer les citation avec "/footnotemark"
\nocite{*}

%choix du style de la biblio
\bibliographystyle{apalike} 
\renewcommand\bibname{Références}
\bibliography{bibliographie}
\begin{appendix}
\chapter{La reconnaissance des empreintes digitales et palmaires} % Main appendix title
\lhead{ANNEXE A.}
\newcommand{\hbAppendixPrefix}{A.}
%
\renewcommand{\thefigure}{\hbAppendixPrefix\arabic{figure}}
\setcounter{figure}{0}
\renewcommand{\thetable}{\hbAppendixPrefix\arabic{table}} 
\setcounter{table}{0}
\renewcommand{\theequation}{\hbAppendixPrefix\arabic{equation}} 
\setcounter{equation}{0}
\label{Appendix1} % Change X to a consecutive letter; for referencing this appendix elsewhere, use \ref{AppendixX}


\section{Capture des empreintes}
La capture est la première phase dans la reconnaissance, l’utilisateur pose ou passe son index (ou plus rarement son pouce) ou sa main sur la surface active du système de capture. Généralement, il existe deux méthodes pour collecter une empreinte, une méthode hors-ligne ou en ligne :
\begin{itemize}
	\item \textbf{Acquisition hors-ligne :} obtenir une empreinte encrée sur un papier et la scanne par un scanner.
	\item \textbf{Acquisition en ligne :} capturer l'empreinte  directement en utilisant un capteur d'empreinte.
\end{itemize}
La première méthode a été largement utilisée. Cependant, avec le développement des capteurs, récemment, sont de plus en plus utilisés dans la collecte des empreintes \citep{Bhanu2004}, les types de capteurs existants sur le marché se divisent en trois grandes familles de capteurs : les capteurs optiques, les capteurs thermoélectriques et les capteurs échographiques.
\section{Prétraitement}
\label{pretrait}
C’est une phase essentielle avant d’effectuer l’étape d’extraction des caractéristiques dans la reconnaissance dont l’objectif est de réduire le bruit et les différentes altérations des empreintes capturées, sans affecter la structure globale et locale des crêtes et des vallées du système  \citep{hong1998integrating}. Le prétraitement des empreintes comprend cinq étapes : la segmentation, la normalisation, le filtrage, la binarisation, et la squelettisation. 
\subsection{Segmentation}
Les images sont décrites par deux régions les images du premier plan et d’arrière-plan. La segmentation consiste à séparer des régions du premier plan qui contient les crêtes et les vallées. Ces régions sont souvent appelées la région d'intérêt (RoI :Region Of Interest) \citep{Babatunde2012}. Elle permet de limiter la région à traiter par élimination des régions en dehors des bords de la ROI où se trouve le bruit introduit pendant l’acquisition des images, Donc réduire le temps de traitement et l'extraction de fausses caractéristiques. Une segmentation correcte peut être, dans certains cas, très difficile, en particulier dans une image d'empreinte  de mauvaise qualité ou des images bruitées, comme la présence de latents. 
\begin{center}
	\begin{figure}[H]
		\centering
			\fbox{\includegraphics[width=0.55\linewidth]{Resources/fingersegmented}}
		\caption{Empreinte digitale segmentée.}
		\label{fig:annexefingersegmented}
	\end{figure}
\end{center}
\subsection{Normalisation}
La normalisation a un objectif d’améliorer la qualité de l’image en éliminant le bruit et en corrigeant les déformations de l'intensité de l'image causée lorsque l’utilisateur pose son doigt incorrectement pendant l’acquisition d’image d’empreinte digitale.
\subsection{Filtrage}
L'image d'empreinte  normalisée sont filtrées pour éliminer le bruit et les caractéristiques parasites d’empreinte, et également pour conserver les véritables structures de crêtes et de vallées.
\begin{center}
	\begin{figure}[H]
		\centering
		\includegraphics[width=0.55\linewidth]{Resources/fingerfiltred}
		\caption{Filtre appliqué sur un empreinte digitale \citep{saveski2010development}.}
		\label{fig:annexefingerfiltred}
	\end{figure}
\end{center}
\subsection{Binarisation}
Les images en 256 niveaux de gris seront transformées en des images binaires de deux niveaux (noir ou blanc). Les pixels en noir correspondent aux crêtes et les pixels en blanc correspondent aux vallées. La binarisation peut être classée en deux catégories de seuillage : 
\begin{itemize}
	\item \textbf{Seuillage globale :} un seul seuil est utilisé dans toute l’image.
    \item \textbf{Seuillage local :} les valeurs des seuils sont déterminées localement, pixel par pixel ou bien région par région. 
    
\end{itemize}
 Le seuillage consiste à comparer les niveaux de gris d'une image avec un seuil pré-calculé pour décider à quelle des deux classes appartient ce point.
\begin{center}
	\begin{figure}[H]
		\centering
		\fbox{\includegraphics[width=0.55\linewidth]{Resources/fingerbinar}}
				    \captionsetup{justification=centering}
		\caption{a). Image avant la binarisation, b). Binarisation avec un seuillage local, c). Binarisation avec un seuillage global \citep{saveski2010development}.}
		\label{fig:annexefingerbinar}
	\end{figure}
\end{center}
\subsection{Squelettisation}
La squelettisation est une procédure qui s’effectue sur l’image binaire pour réduire l’épaisseur des lignes à 1 pixel, tout en conservant la connexité des crêtes (c'est-à-dire que la continuité des crêtes doit être respectée, il ne faut pas introduire de trous).
\begin{center}
	\begin{figure}[H]
		\centering
		\fbox{\includegraphics[width=0.45\linewidth]{fingersquel}}
		    \captionsetup{justification=centering}
		\caption{La squelettisation d'une image binaire : a). L’image avant la squelettisation, b). L’image après la squelettisation \citep{maltoni2009handbook}.}
		\label{fig:annexefingersquel}
	\end{figure}
\end{center}
\section{Post-traitement}
\label{posttrait}
Une étape d'élimination des fausses minuties qui peuvent être détectées pendant l'étape d'extraction, afin d'améliorer les résultats de la phase d'appariement et d'augmenter les performances de la reconnaissance. Il existe plusieurs types de fausses minuties, dont : les segments trop courts, les branches parasites, les ponts, …etc (voir figure \ref{fig:annexfalse}).
\begin{center}
	\begin{figure}[H]
		\centering
		\includegraphics[width=0.6\linewidth]{Resources/annexfalse}
		\captionsetup{justification=centering}
		\caption{Les types de fausses minuties et les résultats d'élimination.\citep{maltoni2009handbook}.}
		\label{fig:annexfalse}
	\end{figure}
\end{center}
 
\section{Extraction de la carte d'orientations OM }
\label{carteOM}
Consiste à obtenir une représentation du flux de crête locale pour chaque sous-région (bloc) dans l'empreinte digitale. Pour une image donné de $ N*M $ pixels et des blocs d'orientation de dimension de $ n*m $ , la carte d'orientations OM est une matrice de dimension de $  N/n * M/m $ dans les éléments sont les angles en radians qui appartient à l'intervalle. 

\section{Transformée de Hough}
\label{Hough}
Transformée de HoughLa transformée de Hough est utilisée pour détecter la présence de relations structurelles spécifiques entre des pixels dans une image. Hough propose une méthode de détection basée sur une transformation d’image permettant la reconnaissance de structures simples (droite, cercle, ...) liant des pixels entre eux. Pour limiter la charge de calcul, l’image originale est préalablement limitée aux contours des objets puis binarisée (2 niveaux possibles pour coder l’intensité pixel) \citep{bergounioux2008quelques}.\\
Cependant les méthodes qui utilisent cette transformation étaient limitées aux images de bord binaire. Pour cela une méthode généralisés introduite par \citep{bergounioux2008quelques}, qui permet à détecter certaines courbes analytiques dans les images en niveau de gris, en particulier les lignes, cercles et les paraboles. Le cas de détection de lignes est le plus exploité dans l’appariement des minuties.
\section{Classification}
Nous présentons une autre méthode de classification appliquée sur l'empreinte palmaire sous sa représentation spectrale :
\subsection{Filtre à corrélations}
\label{filterPalm}
Cette méthode de classification propose d’appliquer plusieurs filtres de corrélations par classe (une classe par paume de main d’un individu) afin d’améliorer la précision de la reconnaissance.
Un filtre de corrélation est un filtre à deux classes qui produit un pic aigu s’il a en entrée sa classe (i.e. si l’image et le filtre sont corrélés) et un bruit sinon (voir les figures \ref{fig:chapitre3filtrecorrelation1} et \ref{fig:chapitre3filtrecorrelation2}) \citep{hennings2007palmprint}.
\begin{center}
	\begin{figure}[H]
		\centering
		\includegraphics[width=0.7\linewidth]{chapitre3filtrecorrelation1}
		\caption{ Sortie du filtre de corrélation au cas d’une image d’empreinte palmaire authentique.}
		\label{fig:chapitre3filtrecorrelation1}
	\end{figure}
\end{center}
`\begin{center}
	\begin{figure}[H]
		\centering
		\includegraphics[width=0.7\linewidth]{chapitre3filtrecorrelation2}
		\caption{Sortie du filtre de corrélation au cas d’une image d’empreinte palmaire non authentique.}
		\label{fig:chapitre3filtrecorrelation2}
	\end{figure}
\end{center}
\section{Extraction des caractéristiques avec DCT}
\label{dct}
Généralement, les images sont décomposées en blocs et sur lesquelles des transformations peuvent être effectuées. La Transformée en cosinus discrète (DCT) convertit les variations spatiales en variations de fréquence en calculant des coefficients de DCT, ces coefficients peuvent être utilisés en tant que vecteurs de caractéristiques \citep{bai2013analyse}.
 \section{Extraction des caractéristiques avec SIFT}
 \label{sift}
 La méthodes des SIFT (scale-invariant feature transform= transformation de caractéristiques visuelles invariante à l'échelle), est une méthode développée par
David Lowe en 2004, permettant de transformer une image en ensemble de vecteurs de caractéristiques qui sont invariants par transformations géométriques usuelles (homothétie, rotation) et de manière moins fiables aux transformations affines et à l'illumination \citep{mikolajczyk2005performance}.
 
\end{appendix}

%voir wiki pour plus d'information sur la syntaxe des entrées d'une bibliographie

\end{document}